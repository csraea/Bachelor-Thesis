% !TEX root = ../thesis.tex

\chaptermark{Úvod}
\phantomsection
\addcontentsline{toc}{chapter}{Úvod}

\chapter*{Úvod}

\par Poradie sekvencie DNA a jej variácie sú samotným aspektom, ktorý určuje vývojové procesy organizmu,
určuje náchylnosť na rôzne choroby a jedinečne identifikuje každého tvora. Táto oblasť bola vždy na periférii
záujmov vedeckej spoločnosti, od objavu v roku 1869 biochemikom Fredricha Mieschera narodeného vo Švajčiarsku.
Napríklad Projekt ľudského genómu (HGP), ktorý sa začal 1. októbra 1990 a bol dokončený v apríli 2003, bol jedným z najväčších počinov.
bádania v dejinách vedy. Bolo to zamerané na čítanie všetkých sekvencií DNA nášho druhu, Homo sapiens. Všetko vo všetkom,
HGP nám predstavil schopnosť čítať kompletný genetický plán prírody pre stavbu človeka.
Napriek úspešnému dokončeniu projektu však stále existuje množstvo neznámych vlastností DNA, ktoré si vyžadujú dôkladné štúdium.

Väčšina ľudí nevie, čo je bioinformatika, a táto práca je pokusom o jej ponorenie.

Pandémia COVID-19 priniesla nové výzvy pre ľudstvo a zvlášť pre bioinformatiku: po sekvenovaní by sa mal každý genóm správne analyzovať, aby sa lepšie pochopili jeho vlastnosti.
Počas analýzy genómu sa často používajú rôzne vizualizačné techniky, aby sa dali dáta ľahko pochopiť.
Preto si vizualizácia štruktúry a vlastností genómu SARS-CoV-2 zaslúži osobitnú pozornosť.

Prvá kapitola tejto diplomovej práce je zameraná na analýzu všeobecnej štruktúry genómu rôznych organizmov (eukaryotov, prokaryotov a vírusov) s cieľom pochopiť, ako si ich vizualizovať.
Táto kapitola tiež obsahuje prehľad existujúcich riešení predstavujúcich údaje o genóme.

Druhá kapitola tejto práce obsahuje analýzu súborov, ktoré sa používajú na ukladanie údajov SARS-CoV-2, analýzu a implementáciu rôznych 2D vizuálnych techník do genómu koronavírusu.
Táto práca navyše navrhuje nový prístup, ako je možné vizualizovať údaje o genóme, aby bolo možné vidieť najmenšie rozdiely v genómoch bez dôkladného skúmania.
Druhá kapitola navyše popisuje zloženie softvéru, ktorý je vyrobený zo skriptov, ktoré sa použili na vizualizáciu genómu SARS-CoV-2.

Aj keď existuje niekoľko nástrojov na spracovanie DNA, problém reprezentácie rôznych vlastností genómu, ktoré sa môžu líšiť,
pokiaľ ide o počet konkrétnych génov alebo úplné chromozómov (ak sú prítomné), zostáva stále aktuálny. Navyše,
spracovanie genómu a jeho vizualizácia si vyžaduje efektívny prístup, pokiaľ ide o veľkosť údajov a výpočtové schopnosti
priemerného počítača.
Cieľom tejto práce je reprezentovať niektoré kľúčové vlastnosti genómu takým spôsobom.

\bigskip
\bigskip

{\noindent\LARGE{\textbf{Formulácia úlohy}}}

\smallskip
\smallskip

V tejto bakalárskej práci by som chcel analyzovať štruktúru genómu organizmov, opísať existujúce riešenia predstavujúcich údaje o genóme.
Samotným cieľom je vykonať analýzu, vizualizáciu a komparáciu genómu SARS-CoV-2 pomocou rôznych techník a porovnať získané výsledky s existujúcimi.
Ďalším cieľom je zostaviť všetky skripty, ktoré boli použité na analýzu a vizualizáciu, do samostatnej aplikácie.