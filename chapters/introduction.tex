% !TEX root = ../thesis.tex

\chaptermark{Úvod}
\phantomsection
\addcontentsline{toc}{chapter}{Úvod}

\chapter*{Úvod}

\par The order of DNA sequence and its variations are the very aspect which dictates the developmental processes of an organism, 
determines susceptibility to various diseases and uniquely identifies each creature. This area has always been on the periphery 
of the interests of scientific society, since the discovery in 1869 by Swiss-born biochemist Fredrich Miescher. 
For instance, The Human Genome Project (HGP) which started on October 1, 1990 and completed in April 2003 was one of the greatest feats 
of exploration in history of science. It was aimed at reading all the DNA sequences of our species, Homo sapiens. All in all, 
HGP introduced us the ability to read nature's complete genetic blueprint for building a human being. 
However, despite the successful completion of the project, a number of unknown DNA properties is still exists and demands the profound studying.

The most of the people do not know what bioinformatics is and this thesis is an attempt of diving in it.

The COVID-19 pandemic intoduced new challenges for the humanity, and bioinformatics particularly: after the sequencing every genome should be analyzed properly in order to obtain better understanding of its features.
Often, during the genome analysis different visualization techniques are used in order to portray data in an easy-to-understand manner.
Therefore, the visualization of SARS-CoV-2 genome structure and properties deserves special attention.

The first chapter of this thesis is aimed at analyzing the general genome structure of different organisms (eukaryotes, prokaryotes and viruses) in order to understand how to visualize it. 
Also, this chapter contains the overview of existing solutions for genome data representing.

The second chapter of this thesis contains the analysis of files that are used to store SARS-CoV-2 data, analysis and implementaton of different 2D visualiaton techniques to the coronavirus genome.
Moreover, this thesis suggests a new approach of how genome data can be visualized in order to see the smallest differences in genomes without profound investigation.
In addition, the second chapter also describes the composition of the software that is made of the scripts that were used to visualize SARS-CoV-2 genome.

Although several DNA processing tools exist, the problem of representing different genome properties which might vary at various species, 
concerning either the number of particular genes or complete chromosomes (if they are present), remains still actual. Moreover, 
the processing of the genome and its visualization demand an efficient approach, concerning the size of data and computational capabilities 
of an average computer. This work aims at representing some key genome properties in such a way.

\bigskip
\bigskip

{\noindent\LARGE{\textbf{Formulacia ulohy}}}

\smallskip
\smallskip

In this bachelor thesis I would like to analyze the genome structure of organisms, estimate the existing solutions to the genome data representing.
The very aim is to perform SARS-CoV-2 genome analysis, visualization and comprasion using different techniques and to compare the results with the existing solutions.
Another goal is to compose all the scripts that were used for the analysis and visualization into a stand-alone aplication.