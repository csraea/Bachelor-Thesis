% !TEX root = ../thesis.tex

\chaptermark{Úvod}
\phantomsection
\addcontentsline{toc}{chapter}{Úvod}

\chapter*{Úvod}

Sekvencia DNA a jej variácie sú samotným aspektom, ktorý určuje vývojové procesy organizmu, určuje náchylnosť na rôzne choroby a identifikuje každého tvora. 
Táto oblasť bola vždy v centre záujmu vedeckej komunity, od objavenia v roku 1869 biochemikom Fredricha Mieschera.
Napríklad Projekt Ľudského Genómu, ktorý sa začal 1 októbra 1990 a bol dokončený v apríli 2003, bol jedným z najväčších počinov bádania v dejinách vedy. 
Projekt bol zameraný na určenie všetkých sekvencií DNA nášho druhu Homo sapiens. 

V konečnom dôsledku, Projekt Ľudského Genómu umôžnil nám čítanie kompletného genetického plánu prírody pre stavbu človeka.
Avšak čítanie genetického kódu neznamená pochopenie jeho štruktúry, a preto, napriek úspešnému dokončeniu projektu stále existuje množstvo neznámych vlastností DNA, ktoré si vyžadujú dôkladné štúdium.

Pandémia COVID-19 priniesla nové výzvy pre ľudstvo a upriamila pozornosť vedeckej komunity na bioinformatiku: po sekvenovaní by sa mal každý genóm byť správne analyzovaný, aby jeho vlastnosti sa boli pochopiteľné.
Počas analýzy genómu sa často používajú rôzne vizualizačné techniky, aby sa dalo rozlíšiť rôzne vzory, vykonať vizuálne porovnanie medzi rôznymi údajmi a poskytnuť informácie zrozumiteľným spôsobom.
Preto si vizualizácia štruktúry a vlastností genómu SARS-CoV-2 zaslúži osobitnú pozornosť.

Prvá kapitola tejto bakalárskej práce je zameraná na analýzu všeobecnej štruktúry genómu rôznych organizmov (eukaryoty, prokaryoty a vírusy) s cieľom najsť vhodne univerzálne spôsoby ich vizualizácie.
Táto kapitola tiež obsahuje prehľad, porovnanie a analýzu niektorých existujúcich programov určených na reprezentáciu údajov o genóme s cieľom pochopiť moderné prístupy ku riešeniu danej problematiky.

Druhá kapitola tejto práce obsahuje analýzu formátov, ktoré sa používajú na ukladanie údajov o genomoch (na príklade SARS-CoV-2), analýzu a implementáciu rôznych techník pre 2D vizualizáciu genómu koronavírusu.
Táto kapitola navyše navrhuje nový spôsob vizualizácie sekvencií genómu, ktorý umôžní identifikovať najmenšie rozdiely v genómoch bez dôkladného skúmania.
Druhá kapitola navyše opisuje štruktúru a zostavenie softvéru, určeného na vizualizáciu genómu SARS-CoV-2 pomocou použitých metód.

Tretia kapitola tejto práce obsahuje komplexný prehľad vykonanej práce vrátane vyhodnotenia výsledkov a dosiahnutia stanovených cieľov, analyzy výhod a nevýhod vyvinutého riešenia na vizualizáciu genómu.

Táto práca je pokusom o využitie aplikovanej bioinformatiky na vizualizáciu štruktúry genómu moderného koronavírusu.

\bigskip
\bigskip

{\noindent\LARGE{\textbf{Formulácia úlohy}}}

\smallskip
\smallskip

Cieľom práce je vykonať analýzu, vizualizáciu a komparáciu genómu SARS-CoV-2 pomocou rôznych techník a porovnať získané výsledky s existujúcimi údajmi.
Ďalším cieľom je zostavenie samostatnej konzolovéj aplikácie pomocou kódu použitého na samotnú analýzu a vizualizáciu genómu SARS-CoV-2.