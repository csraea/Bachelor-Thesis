% !TEX root = ../thesis.tex

\chaptermark{Úvod}
\phantomsection
\addcontentsline{toc}{chapter}{Úvod}

\chapter*{Úvod}

\par The order of DNA sequence and its variations are the very aspect which dictates the developmental processes of an organism, determines susceptibility to various diseases and uniquely identifies each creature. This area has always been on the periphery of the interests of scientific society, since the discovery in 1869 by Swiss-born biochemist Fredrich Miescher. For instance, The Human Genome Project (HGP) which started on October 1, 1990 and completed in April 2003 was one of the greatest feats of exploration in history of science. It was aimed at reading all the DNA sequences of our species, Homo sapiens. All in all, HGP introduced us the ability to read nature's complete genetic blueprint for building a human being. However, despite the successful completion of the project, a number of unknown DNA properties is still exists and demands the profound studying.

The knowledge of the genome structure has significantly increased in the past few decades thanks to the recent developments in the field of advanced analyzing techniques. The Sanger sequencing technology has been traditionally used to elucidate the DNA sequencing information since it was developed in the 1977th. However, it is capable of obtaining sequences of maximum length of 800 base pairs per one operation, which makes the sequencing process much longer and complicated. In spite of development of new sequencing techniques, some technology limits exist. For instance, the human genome in particular presents a number of major obstacles to correct read alignment, due to its size (∼3 GB) and complexity (∼48\% repetitive sequences), as do other plant plant and vertebrate genomes.

In addition, it is impossible to assemble the whole genome sequence of species using the data merely of one individual due to occurrence of the single nucleotide polymorphisms and mutations which affect the precise result. Several sequencing algorithms and searching methods were developed to deal with such issues which are the basis of the bioinformatics. To be precise, the science was developed to deal with the next problems: assembling the complete nucleic acid sequence from the smaller parts, its comparison, analyzing and searching of similarities.

The usual eukaryotic genome consists not only of nuclear DNA, but also of DNA which is isolated from it and belongs to some organelles (mitochondrial mDNA, plastid DNA) that became a part of the cells in the evolution process. To identify key features and determine the exact genes at the complete DNA sequence, to distinguish the segments belonging to particular chromosomes it must visualized in some way. The whole genome might be visualized either as the two dimensional representation of the nucleotide sequence or as the 3D model of the spatial DNA or RNA architecture. The first way allows to analyze each gene and precisely identify each protein that it encodes and to trace the kinship of species, while the second way provides us with the possibility of understanding the inner cellular processes and the very interaction between different enzymes and nucleic acid from the chemical point of view. This work concerns mainly the first method of visualizing sequencing data. 

Although several DNA processing tools exist, the problem of representing different genome properties which might vary at various species, concerning either the number of particular genes or complete chromosomes (if they are present), remains still actual. Moreover, the processing of the genome and its visualization demand an efficient approach, concerning the size of data and computational capabilities of an average computer. This work aims at representing some key genome properties in such a way.
