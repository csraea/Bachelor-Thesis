% !TEX root = ../thesis.tex

\chapter{Syntetická časť}
\section{Existing Genome Browsers}
There are multiple genome browsers available. Some of them are
mentioned here:
\subsection{The UCSC Genome Browser}
It is one of the big players in genomic data visualization. The browser (Kent
et al. 2002) represents annotations as a series of horizontal tracks laid over
genome. Every track can be viewed in different modes such as dense, or
fully expanded or can be hidden. The user can go deeper on the dense track
25
to open it in full mode. There are many scales possible for the track display.
The lowest is a single chromosome and the highest scale is the sequence of
base pairs.
\subsection{The Galaxy Track Browser}
Visual Analytics is the science of using interactive visualizations in order
to support analytic reasoning. The Galaxy Track browser (J. Goecks et al.
2011) overcomes some of the shortcomings of other genomic browser by
using the concept of Visual Analytics. One of them is that the genome
browsers and their analysis tools are not integrated, this makes it tough
to change the parameter value of a tool so as to observe how the change
impacts the tool output in the browser. This can be done multiple times
to tune a tools parameter to obtain a desired output while staying in the
browser.
The Galaxy Track Browser gives freedom to the user to repeatedly change
the parameter’s value and rerun the tool multiple times. Morever, this can
be done interactively because the tool runs on the subset of the data that is
visible to the user. This is useful because users can receive feedback by manipulating data in real time. It provides a multi-resolution support model,
as well using the Galaxy framework provides visualization analysis easy
sharing of the results, all in just a web browser.
\subsection{Trackster}
Trackster (Jeremy Goecks et al. 2012) is another visual analysis environment
based on the Galaxy platform. It is targeted at analyzing the next generation sequencing data subsets by enabling the user to try different analysis
settings. All the outputs can be then visualized together interactively hence
making it easier to compare and inspect for the setting which works the
best. This also reduces the computational time by a large margin. It allows
dynamic integration of tools which are incorporated in the Galaxy framework. The firm coupling of tool settings and visualization enables rapid
tool parameter space exploration and dynamic data filtering.

\section{Parallel coordinates}
Parallel coordinate plots, in the context of gene expression data usually called profile
plots, are a method for visualizing high dimensional data. A point p ∈ R n is drawn on
n parallel axes by placing i-th vertex of a polyline on the position of the i-th axis that
represents the value of p i . A large number of points can be jointly visualized in parallel
coordinates. For interpreting the parallel coordinate plot, the order of the of the parallel
axes must be known. Parallel coordinate are used for a discrete number of dimensions,
like in discrete time series data. For this data, parallel coordinates are especially useful,
as the slope of the polylines is proportional to the difference between two adjacent time
points. The coordinates can also be spaced proportional to the distance between two
time points. Figure 3.3 shows an example of a parallel coordinate plot for time series
data.
Parallel coordinates were introduced in 1959 by Alfred Inselberg (a previous de-
scription of this concept was published by d’Ocagne in 1885) [83]. Since then, par-
allel coordinates have been used in a multitude of applications. An influential paper
of Wegmann [204] demonstrated several use cases and interpretations. it included high-
dimensional geometric objects and cluster visualization, which is one of the most common
applications of parallel coordinates. For this purpose, color is used to indicate cluster
membership. Parallel coordinate plots can be extended by adding additional dimensions,
for example showing statistical properties of the points.
Plotting many points in parallel coordinates can lead to overplotting. To address
this problem, a number of dimension reduction methods have been suggested, e.g. [89].
Alternatively, clusters can be represented by centroids, leaving out all other points.
Using semitransparent lines gives a better overview of the density of lines in a plot. The
number of dimensions in a parallel coordinate plot is not generally limited. However, a
large number of dimensions might lead to tightly spaced coordinates, which can be hard
to read. For time series, an aspect ratio that causes the average slope of a line segment
to be 45° is considered optimal [74]. This might lead to a trade-off between size and
readability.

\section{Visual Analytics}
While the roots of exploratory data analysis were based on manual calculations and
hand drawn graphics, possibly enhanced by desk calculators, modern methods greatly
increased the speed and handling of visualization and exploratory statistics. Interactivity
28
3.2. Visualization Plots
is a further important aspect made feasible by computer-based visualization. In the same
time, however, the size and complexity of datasets increased even faster then the analysis
tools were improved [95]. New concepts for making sense of large, noisy and heterogenous
datasets are required. Visual analytics makes use of interactive visualization to support
human cognition to analyze and interpret data. The focus lies on the optimal support
of human cognition, which is considered a powerful tool. For this purpose, methods and
results from various scientific disciplines are integrated, including computer graphics,
psychology, cognitive sciences and design.
The overall process of visual analytics can be summarized by the visual analytics
mantra: “Analyse First - Show the Important - Zoom, Filter and Analyse Further - Details on Demand” [95]. It names some of the tools and strategies used in visual analytics.
Analysis methods are used to prepare and filter the data to first visualize concentrating on important features [95]. Visualizations are optimally maximizing data density and
should allow to easily identify patterns and relationships. Commonly tools used for EDA
are applied [27]. Based on an existing visualization, refinements are interactively made:
zooming to get a view that is more coarse or fine, filtering to remove irrelevant items and
further analyses. Details on objects are shown interactively on demand. This process is
iteratively repeated. Each iteration is aimed at providing a useful visual representation
that allows the viewer to make more sense of the data.
As visual analytics is concerned with extremely large datasets, several challenges exist.
The limited space on visual media, especially screens calls for scalable visualizations (and
larger screens) [96]. Analyzing high-throughput stream data can address data storage
problems. Another challenge is the analysis of heterogenous datasets, which arise in
many fields, including systems biology. Automatization of processes, decision support
and evaluation of existing processes are also fields of research in visual analytics.

\label{methodology}

\begin{thebibliography}{9}
    \bibitem{latexcompanion} 
    Stephan Symons, Christian Zipplies, Florian Battke, and Kay Nieselt.
    \textit{tephan Symons, Christian Zipplies, Florian Battke, and Kay Nieselt. Integrative
    Systems Biology Visualization with Mayday. Journal of Integrative Bioinformatics
    2010, 7:3}. 
    Addison-Wesley, Reading, Massachusetts, 1993.
    
    \bibitem{einstein} 
    Matthias Zschunke, Katrin Deubel, Stephan Symons, Janko Dietzsch, and Kay Nieselt. FAGE and VEGA
    \textit{A versatile algorithm and software for resequencing microarrays}. (German) 
    [\textit{On the electrodynamics of moving bodies}]. 
    Annalen der Physik, 322(10):891–921, 1905.

    \bibitem{einstein} 
    Florian Battke, Stephan Symons, Michael Piechotta, Philipp Bruns, Karin Zimmermann, and Kay Nieselt.
    \textit{Integrated    Expression Analysis}. (German) 
    [\textit{On the electrodynamics of moving bodies}]. 
    Annalen der Physik, 322(10):891–921, 1905.
    
    