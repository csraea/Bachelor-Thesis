% !TEX root = ../thesis.tex


\thispagestyle{empty}
	\begin{center}
		\vspace*{1cm}
		
		\textbf{\large Technicka Univerzita v Košiciach }
		
		\vspace{0.4cm}
		Fakulta elektrotechniky a informatiky
		
		\vspace{4.5cm}
		
		\textbf{\Large Dokumentácia k programu na vizualizáciu štruktúry genómu\\}
		\vspace{1.5cm}
		Príloha A k bakalárskej práci
		\vfill
		
		
		\vspace{2.8cm}
		
		{\raggedleft\vfill{%
		 		
			}\par}
		
		{\raggedright\vfill{%
				2021 \quad\quad\quad\quad\quad\quad\quad\quad\quad\quad\quad\quad\quad\quad\quad\quad\quad\quad\quad\quad\quad\quad\quad Oleksandr Korotetskyi
			}\par}
		
		
		
	\end{center}

\newpage
\rhead{Dokumentácia}
\addcontentsline{toc}{chapter}{Documentácia}
\subsubsection{\Large{Úvod}}
\addcontentsline{toc}{section}{Úvod}
Túto dokumentáciu možno považovať za používateľskú príručku k použitiu programu a ako systémovú príručku, ktorá popisuje funkčnosť a architektúru programu.

Dokumentácia je rozdelená na niekoľko častí, ktoré sa venujú zodpovedajúcim témam: inštalácia, scenáre spustenia a použitia, architektúra aplikácie.

\subsubsection{\Large{Inštalácia}}
\addcontentsline{toc}{section}{Inštalácia}
Aplikácia bola vyvinutá pre použitie hlavne na platformách Unix / Linux, a preto môže pokus o jej inštaláciu na platformu Windows viesť k neočakávanému správaniu programu.


\subsubsection{Python 3.8}
Aplikácia je napísaná v Pythone 3.8, a preto on je nevyhnutný pre použitie programu.
V systémoch založených na Debiane je možné ho nainštalovať pomocou nasledujúcich príkazov v termináli (štandardnóm príkazovóm riadku):
\begin{lstlisting}[language=bash]
  $ sudo apt-get update
  $ sudo apt-get install python3.8
\end{lstlisting}

Pre systémy založené na Fedore by sa mal použiť nasledujúci príkaz:
\begin{lstlisting}[language=bash]
  $ sudo dnf install python3
\end{lstlisting}


\subsubsection{Pip}
Ďalším krokom je inštalácia {\fontfamily{lmtt}\selectfont pip} (správca balíkov python), ktorý sa použije na inštaláciu ďalších závislostí.
Je to môžne urobiť pomocou nasledujúceho príkazu pre distribúcie založené na Debiane:
\begin{lstlisting}[language=bash]
  $ sudo apt-get install python3-pip
\end{lstlisting}

A pre systémy založené na Fedore musia byť použité:
\begin{lstlisting}[language=bash]
  $ curl "https://bootstrap.pypa.io/get-pip.py" -o "get-pip.py"
  $ python get-pip.py
\end{lstlisting}


\subsubsection{Knižnice}
Softvér pracuje na populárnych knižniciach, ktoré poskytujú používateľovi rozšírené funkcie pre sprácovanie genomov a ine účely.
Aby bolo možné aplikáciu používať, je potrebné nainštalovať nasledujúcich 12 balíkov:
{\fontfamily{lmtt}\selectfont
\begin{itemize}
    \item bio==0.4.1
    \item biopython==1.78
    \item matplotlib==3.4.2
    \item numpy==1.19.5
    \item pandas==1.2.4
    \item pillow==8.2.0
    \item pyparsing==2.4.7
    \item requests==2.25.1
    \item seaborn==0.11.1
    \item urllib3==1.26.4
    \item bcbio-gff
    \item dna\_features\_viewer
\end{itemize}
}

Úplný zoznam požadovaných balíkov sa nachádza v súbore \textbf{\fontfamily{lmtt}\selectfont requirements.txt}, ktorý je umiestnený v koreňovom adresári programu.

Samotnú inštaláciu knižníc je možné vykonať v termináli v koreňovom adresári programu pomocou jedného z nasledujúcich príkazov, ktoré sa môžu líšiť v závislosti od systému:
\begin{lstlisting}[language=bash]
  $ pip install -r requirements.txt
\end{lstlisting}
\begin{lstlisting}[language=bash]
  $ pip3 install -r requirements.txt
\end{lstlisting}

\addcontentsline{toc}{section}{Popis spustenia a činnosti aplikácie}
\subsubsection{\Large{Popis spustenia a  činnosti aplikácie}}
Tento vizualizačný nástroj podporuje dva režimy vykonávania: \textit {verbose} a \textit {quiet}.
Aplikácia v režime \textit{verbose} poskytuje používateľovi komentáre a prostredie na triviálnu interakciu, zatiaľ čo režim \textit{quit} je vhodnejší na účely rýchlejšej vizualizácie a automatizácie.

Oba režimy majú rovnaké funkcie, a preto sa líšia iba v tom, ako používateľ povie programovi, čo má robiť.
V režime \textit{verbose} používateľ komunikuje s programom prostredníctvom vstupu a výstupu konzoly, zatiaľ čo v režime \textit{quit} používa iba argumenty príkazového riadku.

Táto časť popisuje rôzne scenáre vykonávania programu v režime \textit{verbose} a sprevádzané príkazom na vykonanie rovnakej akcie iba pomocou argumentov príkazového riadku v režime \textit{quiet}.

Na spustenie aplikácie v režime \textit{verbose} by sa mal použiť jeden z nasledujúcich príkazov:
\begin{lstlisting}[language=bash]
  $ python3 Main.py
\end{lstlisting}
\begin{lstlisting}[language=bash]
  $ python3 Main.py -m v
\end{lstlisting}

Po spustení aplikácie sa objavi hlavné menu nástroja:
\begin{lstlisting}[language=bash]
  +----------------------------------+
  |--- Welcome to the Visualizer! ---|
  +----------------------------------+
  Choose the option: 
  1. Download SARS-CoV-2 genome sequence & associated files
  2. Plot sequence statistics
  3. Gates' visualization
  4. 2D Matrix visualization
  5. Improved 2D Matrix visualization
  6. Plot ORFs
  7. Compare genomes
  8. Exit
  Choice: 
\end{lstlisting}
Užívateľ je schopný voliť rôzné možnosti zadaním zodpovedajúceho im čísla.
Na ukončenie práci s nástrojem je potrebné stlačiť zadať „q“ pri hoci akom vstupe, na ktorý program čaká.


\subsubsection{Scenár 1}
Po výbere prvej možnosti sa v prípade úspechu zobrazia nasledujúce správy.
\begin{lstlisting}[language=bash]
  Necessary files are being downloaded...
  Done!
\end{lstlisting}
Všetky potrebné súbory (SARS-CoV-2.fasta a SARS-CoV-2.gb) sa úspešne stiahli.
Všetky ostatné súbory na vizualizáciu genómom musí používateľ pridať ručne do adresára {\fontfamily{lmtt}\selectfont data}.

\bigskip
Rovnaký scenár je možné vykonať v režime \textit{quiet} bez akýchkoľvek programových správ pomocou nasledujúceho príkazu:
\begin{lstlisting}[language=bash]
  $ python3 Main.py -m q -d
\end{lstlisting}


\subsubsection{Scenár 2}
Po výbere druhej možnosti program požiada používateľa, aby si vybral postupnosť z tých, ktoré sa nachádzajú v adresári {\fontfamily{lmtt}\selectfont data}.
\begin{lstlisting}[language=bash]
  Choose the sequence to plot the statistics of:
  1. alteromonas.fasta
  2. SARS-CoV-2.fasta
  3. ebola.fasta
  Choice: 2
\end{lstlisting}
Ďalším krokom je určenie intervalu, o ktorom sa musia štatistické údaje zobrazovať:
\begin{lstlisting}[language=bash]
  Specify the interval (0 for the entire genome)
  Start:	1223
  End:	0
\end{lstlisting}

Nuly predstavujú defaultné hodnoty, zatiaľ čo program vytvára tento výstup:
\begin{lstlisting}[language=bash]
  Frequencies of nucleotides on the interval [1223;29903]:
  A:	8626
  T:	9251
  G:	5577
  C:	5226
  Total:	28680
  GC-content on interval [1223;29903]:	0.3767%
\end{lstlisting}
Zobrazuju sa základné štatistické údaje analyzovanej sekvencie, ktoré by mohli byť užitočné.


\bigskip
Rovnaký scenár je možné vykonať v režime \textit{quiet} pomocou nasledujúceho príkazu:
\begin{lstlisting}[language=bash]
  $ python3 Main.py -m q -s --pos 1223 0
\end{lstlisting}



\subsubsection{Scenár 3}
Po výbere tretej možnosti program požiada používateľa, aby si vybral postupnosť z tých, ktoré sa nachádzajú v adresári {\fontfamily{lmtt}\selectfont data}.
\begin{lstlisting}[language=bash]
  Choose the sequence to visualize using Gates' method:
  1. alteromonas.fasta
  2. SARS-CoV-2.fasta
  3. ebola.fasta
  Choice: 3
\end{lstlisting}
Ďalším krokom je určenie intervalu sekvencie genómu, ktorý sa bude vizualizovať:
\begin{lstlisting}[language=bash]
  Specify the interval (0 for the entire genome)
  Start:	0
  End:	8000
  Done!
\end{lstlisting}

Generovaný obrázok je uložený v adresári {\fontfamily{lmtt}\selectfont out}.

\bigskip
Rovnaký scenár je možné vykonať v režime \textit{quiet} pomocou nasledujúceho príkazu:
\begin{lstlisting}[language=bash]
  $ python3 Main.py -m q -g ebola.fasta --pos 0 8000
\end{lstlisting}

\subsubsection{Scenár 4}
Po výbere štvrtej možnosti program požiada používateľa, aby si vybral postupnosť z tých, ktoré sa nachádzajú v adresári {\fontfamily{lmtt}\selectfont data}.
\begin{lstlisting}[language=bash]
  Choose the sequence to visualize using 2D Matrix method:
  1. alteromonas.fasta
  2. SARS-CoV-2.fasta
  3. ebola.fasta
  Choice: 3
  Done
\end{lstlisting}

Táto vizualizácia nepodporuje voľbu intervalu a vygenerovaný obrázok sa ukláda do adresára {\fontfamily{lmtt}\selectfont out}.

\bigskip
Rovnaký scenár je možné vykonať v režime \textit{quiet} pomocou nasledujúceho príkazu:
\begin{lstlisting}[language=bash]
  $ python3 Main.py -m q -x ebola.fasta
\end{lstlisting}

\subsubsection{Scenár 5}
Po výbere piatej možnosti program požiada používateľa, aby si vybral postupnosť z tých, ktoré sa nachádzajú v adresári {\fontfamily{lmtt}\selectfont data}.
\begin{lstlisting}[language=bash]
  Choose the sequence to visualize using 2D HMatrix method:
  1. alteromonas.fasta
  2. SARS-CoV-2.fasta
  3. ebola.fasta
  Choice: 2
\end{lstlisting}

Ďalším krokom je zadavánie veľkosti vytvoreného obrázka a po ňom program vygeneruje nasledujúci výstup:
\begin{lstlisting}[language=bash]
  Specify the size of image (preferably a power of 2, >= 512):
  Size (px):	2048
  seed: f9fa11164acc370f5c187a286c25dcffe0b93363c68ce5d658d83e
  w, h: 1024.0, 512.0
  w, h: 256.0, 256.0
  opacity: 50%
  w, h: 128.0, 64.0
  opacity: 37%
  w, h: 32.0, 32.0
  opacity: 25%
  w, h: 16.0, 8.0
  opacity: 15%
  w, h: 4.0, 4.0
  opacity: 9%
  w, h: 2.0, 1.0
  opacity: 5%
  Done
\end{lstlisting}

Vygenerovaný obrázok sa ukláda do adresára {\fontfamily{lmtt}\selectfont out}.

\bigskip
Rovnaký scenár je možné vykonať v režime \textit{quiet} bez výstupu konzoly pomocou nasledujúceho príkazu:
\begin{lstlisting}[language=bash]
  $ python3 Main.py -m q -i ebola.fasta -S 2048
\end{lstlisting}



\subsubsection{Scenár 6}
Po výbere šiestej možnosti program požiada používateľa, aby si vybral anotačný súbor genómu, ktorý sa má použiť, z tých, ktoré sa nachádzajú v adresári {\fontfamily{lmtt}\selectfont data}.
\begin{lstlisting}[language=bash]
  Choose the annotation file to visualize ORFs:
  1. ebola.gb
  2. SARS-CoV-2.gbk
  Choice: 2
  Done  
\end{lstlisting}
Vygenerovaný obrázok sa úspešne ulkláda do adresára {\fontfamily{lmtt}\selectfont out}.

\bigskip
Rovnaký scenár je možné vykonať v režime \textit{quiet} bez výstupu konzoly pomocou nasledujúceho príkazu:
\begin{lstlisting}[language=bash]
  $ python3 Main.py -m q -o SARS-CoV-2.gbk
\end{lstlisting}



\subsubsection{Scenár 7}
Po výbere siedmej možnosti program požiada používateľa, aby vybral sekvencie genómu na porovnanie so sekvenciami, ktoré sa nachádzajú v adresári {\fontfamily{lmtt}\selectfont data}.
\begin{lstlisting}[language=bash]
  Choose the first sequence to compare:
  1. alteromonas.fasta
  2. SARS-CoV-2.fasta
  3. ebola.fasta
  Choice: 2
  Choose the second sequence to compare:
  1. alteromonas.fasta
  2. SARS-CoV-2.fasta
  3. ebola.fasta
  Choice: 3
  Similarity (%): 72  
\end{lstlisting}
Po samotnom porovnaní program zadá percento podobnosti.

\bigskip
Rovnaký scenár je možné vykonať v režime \textit{quiet} bez výstupu konzoly pomocou nasledujúceho príkazu:
\begin{lstlisting}[language=bash]
  $ python3 Main.py -m q -c SARS-CoV-2.fasta ebola.fasta
\end{lstlisting}


\subsubsection{Použitie režimu Quiet}
Pre ziskanie dokladnéj informácií o režime \textit{quiet}, používateľ môže zadať nasledujúci príkaz:
\begin{lstlisting}[language=bash]
  $ python3 Main.py -h
\end{lstlisting}

Tento príkaz zobrazuje všetky podporované argumenty príkazového riadku:
\begin{lstlisting}[language=bash]
usage: Main.py [-h] [-m {q,v}] [-d] [-g GATES] [-o ORF] [-s] 
               [-x MATRIX] [-i HASH] [-c COMP COMP] [-S SIZE]
               [-p POS POS] [-a] [-n NAME]

optional arguments:
  -h, --help            show this help message and exit
  -m {q,v}, --mode {q,v}
                        Execution mode: quiet / verbose
  -d, --download        Download SARS-CoV-2 genome associated 
                        files
  -g GATES, --gates GATES
                        Perform Gates' visualization. Parameter
                        is an input sequence filename
  -o ORF, --orf ORF     Plot ORFs of the genome. Parameter is 
                        an input sequence filename
  -s, --stat            Obtain genome statistical data including
                        the distribution of nucleotides 
                        and a GC-content
  -x MATRIX, --matrix MATRIX
                        Plot the nucleotide sequence into a 2D 
                        matrix. Parameter is the input sequence 
                        filename
  -i HASH, --hmatrix HASH
                        Plot the nucleotide sequense into Hashed
                        2D matrix. Parameter is input sequence 
                        filename
  -c COMP COMP, --compare COMP COMP
                        Compare specified genome sequences using 
                        the pairwise2 algorithm
  -S SIZE, --size SIZE  Size of the picture side in pixels
  -p POS POS, --pos POS POS
                        Start and end positions of the nucleotide 
                        sequence to perform an action (0 for default)
  -a, --all             Perform all possible actions but comparison
                        in the default mode
  -n NAME, --name NAME  Input sequnce filename
\end{lstlisting}


\rhead{Dokumentácia}
\subsubsection{\Large{Architektúra aplikácie}}
\addcontentsline{toc}{section}{Architektúra aplikácie}
Vyvinutý program predstavuje samostatnú konzolovú aplikáciu, ktorá sa skladá z 8 modulov umiestnených v koreňovom adresári programu.
Obsah adresára je uvedený nižšie:
\textbf{\fontfamily{lmtt}\selectfont
\begin{itemize}
    \item Main.py
    \item Comparison.py
    \item GatesVisualization.py
    \item MatrixVisualization.py
    \item HMatrixVisualization.py
    \item ORFPlotter.py
    \item SeqCollector.py
    \item StatGenerator.py
    \item requirements.txt
\end{itemize}
}

Počas chodu programu sa vytvárajú dva ďalšie adresáre so súbormi, ak neexistujú: \textbf{\fontfamily{lmtt}\selectfont data} a \textbf{\fontfamily{lmtt}\selectfont out}.

Prvý adresár obsahuje stiahnuté sekvencie a program ho považuje za zdrojový adresár všetkých sekvencií genómu a súborov anotácií genómu, s ktorými program pracuje.
Preto, aby bolo možné vizualizovať a pracovať s vlastnými genómami, ich súbory musia byť vložené do adresára \textbf{\fontfamily{lmtt}\selectfont data}.

Druhý slúži na uloženie všetkých výstupných obrázkov formátu {\fontfamily{lmtt}\selectfont .png}, ktoré program vytvorí.
Preto, aby si užívateľ mohol pozrieť vykonané vizualizácie, musí ich vyhľadať v adresári \textbf{\fontfamily{lmtt}\selectfont out}.

\rhead{Dokumentácia}
\subsubsection{Modules description}
\textbf{\fontfamily{lmtt}\selectfont Main.py} je hlavný modul programu, ktorý je zodpovedný za použitie zvyšných modulov na vykonanie zadanej úlohy.
Zaoberá sa vstupom a výstupom z konzoly, navrhuje dostupné metódy vizualizácie a získava podrobnosti potrebné na ich výkon.
Obsahuje nasledujúce funkcie:
\begin{itemize}
  \item \textbf{\fontfamily{lmtt}\selectfont verifyArgs()} -- overuje a kontroluje argumenty príkazového riadku, ak je program spustený v režime \textit{quiet}. Ak sa vyskytne chyba, program sa zastaví.
  \item \textbf{\fontfamily{lmtt}\selectfont welcomeBanner()} -- ak je zapnutý režim \textit{verbose}, zobrazí "uvítací banner".
  \item \textbf{\fontfamily{lmtt}\selectfont mainMenu()} -- ak je režim \textit{verbose} zapnutý, zobrazí hlavné menu aplikácie a požiada používateľa, aby vybral možnosť pokračovania; skontroluje vstup používateľa. Vráti číslo vybratého scenára.
  \item \textbf{\fontfamily{lmtt}\selectfont vObtainFiles(msg)} -- ak je zapnutý režim \textit{verbose}, vypíše všetky súbory vo formáte {\fontfamily{lmtt}\selectfont FASTA} a požiada používateľa, aby si jeden vybral. Parameter {\fontfamily{lmtt}\selectfont msg} predstavuje správu, ktorá sa má zobraziť. Vráti názov vybraného súboru, ak je prítomný, v opačnom prípade sa zobrazí chybové hlásenie a funkcia vráti hodnotu {\fontfamily{lmtt}\selectfont None}.
  \item \textbf{\fontfamily{lmtt}\selectfont vObtainFiles2(msg)} -- ak je zapnutý režim \textit{verbose}, vypíše všetky súbory vo formáte {\fontfamily{lmtt}\selectfont GenBank} a požiada používateľa, aby si jeden vybral. Parameter {\fontfamily{lmtt}\selectfont msg} predstavuje správu, ktorá sa má zobraziť. Vráti názov vybraného súboru, ak je prítomný, v opačnom prípade sa zobrazí chybové hlásenie a funkcia vráti hodnotu {\fontfamily{lmtt}\selectfont None}.
  \item \textbf{\fontfamily{lmtt}\selectfont vObtainInterval()} -- ak je režim \textit{verbose} zapnutý, požiada používateľa, aby určil interval sekvencie genómu, na ktorom má vykonať akciu. Vráti pozície {\fontfamily{lmtt}\selectfont start} a {\fontfamily{lmtt}\selectfont end} po ich overení.
  \item \textbf{\fontfamily{lmtt}\selectfont vObtainSize()} -- ak je zapnutý režim \textit{verbose}, požiada používateľa, aby určil veľkosť pre vygenerovanie štvorcového obrázku. Vráti veľkosť strany obrázka v pixeloch.
  \item \textbf{\fontfamily{lmtt}\selectfont main()} -- vykoná hlavný cyclus programu a určuje, ktorú akciu má vykonať podľa argumentov príkazového riadku a vstupu používateľa. V režime \textit{quiet} končí program po vykonaní akcie, zatiaľ čo v režime \textit{verbose} znova zobrazí hlavné menu aplikácie.
\end{itemize}



\textbf{\fontfamily{lmtt}\selectfont SeqCollector.py} je zodpovedný za stiahnutie všetkých požadovaných sekvencií a súborov anotácií z databázy NCBI pre vizualizáciu genómu SARS-CoV-2.
V tejto chvíli nepodporuje sťahovanie súborov spojených s inými genómami.
Obsahuje nasledujúce funkcie:
\begin{itemize}
  \item \textbf{\fontfamily{lmtt}\selectfont downloadFiles()} -- vytvorí adresár {\fontfamily{lmtt}\selectfont data}, ak neexistuje, a stiahne (vyžaduje sa internetové pripojenie) sekvenciu genómu a anotačné súbory SARS-CoV-2.
\end{itemize}



\textbf{\fontfamily{lmtt}\selectfont StatGenerator.py} získava štatistické údaje, ako je obsah GC a distribúcia nukleotidov / aminokyselín.
Užívateľ si môže zvoliť oblasť genómu, ktorá sa má štatisticky analyzovať.
Obsahuje nasledujúce funkcie:
\begin{itemize}
  \item \textbf{\fontfamily{lmtt}\selectfont getStats(filename, mode, start, end)} -- overuje typ {\fontfamily{lmtt}\selectfont filename}. Overuje {\fontfamily{lmtt}\selectfont start} a {\fontfamily{lmtt}\selectfont end} pozície. Funkcia zastaví vykonávanie programu v chybových prípadoch a zobrazí príslušné chybové hlásenie. Vypisuje štatistické údaje o zadanom intervale sekvencie genómu a poskytne používateľovi ďalšie komentáre v režime \textit{verbose}.
\end{itemize}



\textbf{\fontfamily{lmtt}\selectfont GatesVisualization.py} vykonáva vizualizáciu pomocou Gatesovej metódy do súboru {\fontfamily{lmtt}\selectfont -Gates.png} .
Užívateľ je schopný zvoliť oblasť genómu ktorú chce vizualizovať.
Obsahuje nasledujúce funkcie:
\begin{itemize}
  \item \textbf{\fontfamily{lmtt}\selectfont visualize(filename, mode, start, end)} -- overuje typ {\fontfamily{lmtt}\selectfont filename}. Overuje {\fontfamily{lmtt}\selectfont start} a {\fontfamily{lmtt}\selectfont end} pozície. Funkcia zastaví vykonávanie programu v chybových prípadoch a zobrazí príslušné chybové hlásenie. Vykonáva Gatesovu vizualizáciu určeného intervalu sekvencie genómu.
  \item \textbf{\fontfamily{lmtt}\selectfont save(outFileName, image)} -- vytvorí adresár {\fontfamily{lmtt}\selectfont out} ak neexistuje, a uloží vygenerovaný obrázok {\fontfamily{lmtt}\selectfont outFileName} do adresára {\fontfamily{lmtt}\selectfont out}.
\end{itemize}



\textbf{\fontfamily{lmtt}\selectfont MatrixVisualization.py} kreslí zvolený genóm pomocou generácie 2D matice do súboru {\fontfamily{lmtt}\selectfont -Matrix.png} .
Veľkosť výstupného obrázka sa počíta automaticky.
Obsahuje nasledujúce funkcie:
\begin{itemize}
  \item \textbf{\fontfamily{lmtt}\selectfont visualize(filename)} -- overuje typ {\fontfamily{lmtt}\selectfont filename}. Funkcia zastaví vykonávanie programu v chybových prípadoch a zobrazí príslušné chybové hlásenie. Vykoná vizualizáciu sekvencie špecifikovaného genómu jeho vykreslením do 2D matice.
\end{itemize}



\textbf{\fontfamily{lmtt}\selectfont HMatrixVisualization.py} kreslí genóm do 2D matice vybranej veľkosti pomocou algoritmu hash funkcie do súboru {\fontfamily{lmtt}\selectfont -Hmatrix.png} .
Veľkosť výstupného obrázka môže byť zádana používateľom.
Obsahuje nasledujúce funkcie:
\begin{itemize}
  \item \textbf{\fontfamily{lmtt}\selectfont save(outFileName, image)} -- vytvorí adresár {\fontfamily{lmtt}\selectfont out}, ak neexistuje, a uloží vygenerovaný obrázok{\fontfamily{lmtt}\selectfont outFileName} do adresára {\fontfamily{lmtt}\selectfont out}.
  \item \textbf{\fontfamily{lmtt}\selectfont drawLayer(imgSize, depth, mode)} -- kreslí farébne bloky na základe {\fontfamily{lmtt}\selectfont getRandomColor()} vo veľkosti {\fontfamily{lmtt}\selectfont getBlockSize(imgSize, depth)} pre jednotlivé vrstvy. Vráti obrázok aktuálnej vrstvy. Poskytuje používateľovi ďalšie komentáre v režime \textit{verbose}.
  \item \textbf{\fontfamily{lmtt}\selectfont getHash(filename)} -- počíta hash sekvencie genómu {\fontfamily{lmtt}\selectfont filename}.
  \item \textbf{\fontfamily{lmtt}\selectfont getRandomColor()} -- vráti n-ticu náhodných hodnôt farieb vo formáte RGB.
  \item \textbf{\fontfamily{lmtt}\selectfont getBlockSize(imgSize, depth)} -- počíta a vracia {\fontfamily{lmtt}\selectfont width} a {\fontfamily{lmtt}\selectfont height} podľa {\fontfamily{lmtt}\selectfont imgSize} a {\fontfamily{lmtt}\selectfont depth}. S každou iteráciou cyklu sa každá strana bloku delí na polovicu alebo na štvrtiny, v závislosti od {\fontfamily{lmtt}\selectfont depth}.
  \item \textbf{\fontfamily{lmtt}\selectfont visualize(filename, mode, ssize)} -- skontroluje typ {\fontfamily{lmtt}\selectfont filename} a zastaví vykonávanie programu v prípade chýb. Vykonáva 2D Hashed Matrix vizualizáciu určenej sekvencie genómu rekurzívnym spôsobom. Upraví {\fontfamily{lmtt}\selectfont seed} na generovanie náhodných čísel na základe funkcie {\fontfamily{lmtt}\selectfont getHash(filename)}. Zlúči vrstvy veľkosti {\fontfamily{lmtt}\selectfont size} vytvorené funkciou {\fontfamily{lmtt}\selectfont drawLayer(size, depth, mode)} v závislosti od definovanej {\fontfamily{lmtt}\selectfont opacity}. Poskytuje používateľovi ďalšie komentáre v režime \textit{verbose}.
\end{itemize}



\textbf{\fontfamily{lmtt}\selectfont ORFPlotter.py} generuje obraz distribúcie ORF a pomeru obsahu GC v genóme do súboru {\fontfamily{lmtt}\selectfont -ORFs.png} .
Obsahuje nasledujúce funkcie:
\begin{itemize}
  \item \textbf{\fontfamily{lmtt}\selectfont visualize(filename)} -- overuje príponu {\fontfamily{lmtt}\selectfont filname}. Funkcia zastaví vykonávanie programu v chybových prípadoch a zobrazí príslušné chybové hlásenie. Vykoná vizualizáciu súboru s anotáciami určeného genómu zobrazením ORF.
  \item \textbf{\fontfamily{lmtt}\selectfont save(outFileName, plt)} -- vytvorí adresár {\fontfamily{lmtt}\selectfont out} ak neexistuje, a uloží vygenerovaný plot {\fontfamily{lmtt}\selectfont plt} do adresára {\fontfamily{lmtt}\selectfont out}.
\end{itemize}

\textbf{\fontfamily{lmtt}\selectfont Comparison.py} vykonáva porovnanie zvolenych genómov. Percento podobnosti sa získa na základe algoritmu pairwise2.
Obsahuje nasledujúce funkcie:
\begin{itemize}
  \item \textbf{\fontfamily{lmtt}\selectfont compare(filename1, filename2, mode)} -- overuje typy súborov {\fontfamily{lmtt}\selectfont filename1} a {\fontfamily{lmtt}\selectfont filename2} a zastaví vykonávanie programu v prípade chyby a poskytne používateľovi príslušnú správu. Vykonáva porovnanie vybraných sekvencií genómu pomocou algoritmu pairwise2. Poskytuje používateľovi ďalšie komentáre v režime \textit{verbose}.
\end{itemize}

\subsubsection{\Large{Záver}}
\addcontentsline{toc}{section}{Záver}
Táto dokumentácia predstavuje komplexný prehľad softvéru, ktorý bol vyvinutý počas bakalárskej práce „Vizualizácia štruktúry genómu“.

Aplikácia pracuje v dvoch možných režimoch a umožňuje používateľovi vizualizovať a analyzovať genómy rôznych organizmov pomocou sady vopred určených techník.

Inštalácia, vykonanie, technické aspekty a scenáre použitia boli podrobne popísané v príslušných častiach.

Na záver by sa ďalšie vylepšenia mohli zamerať na aplikovánie objektovo-orientovanej paradigmy programovania na architektúru programu a na pridanie nových funkcionalít.

