%% -----------------------------------------------------------------
%% This file uses UTF-8 encoding
%%
%% For compilation use following command:
%% latexmk -pdf -pvc -bibtex thesis
%%
%% -----------------------------------------------------------------
%%                                     _     _      
%%      _ __  _ __ ___  __ _ _ __ ___ | |__ | | ___ 
%%     | '_ \| '__/ _ \/ _` | '_ ` _ \| '_ \| |/ _ \
%%     | |_) | | |  __/ (_| | | | | | | |_) | |  __/
%%     | .__/|_|  \___|\__,_|_| |_| |_|_.__/|_|\___|
%%     |_|                                          
%%
%% -----------------------------------------------------------------

\documentclass{kithesis}

% Additional packages
\usepackage[main=slovak,english]{babel}

\usepackage{listings}
% Listings settings
% See for details: https://en.wikibooks.org/wiki/LaTeX/Source_Code_Listings
\lstset{
    basicstyle=\small\ttfamily,  % smaller typewriter font
    showstringspaces=false       % don't show spaces in string
}
\def\lstlistingname{Zdrojový kód}

% Variables
%\thesisspec{figures/thesisspec.png} 

\title{My thesis \br (the skeleton)}{Vizualizácia štruktúry genómu\br}

%\author{Janko Hraško}
\author[Bc.]{Oleksandr}{Korotetskyi}[]
\supervisor{doc. Ing. Ján Genči, PhD.} %veduci prace
\consultant{} %konzultant
%\college{University of Žilina}{Žilinská univerzita} %univerzita
%\faculty{Faculty of Electrical Engineering and informatics}{Fakulta elektrotechniky a informatiky} %fakulta
%\department{Department of Computers and Informatics}{Katedra počítačov a informatiky} %katedra
%\departmentacr{DCI}{KPI} % skratka katedry
%\thesis{Master thesis}{Diplomová práca} %typ prace
\submissiondate{28}{5}{2021}
%\fieldofstudy{9.2.1 Informatika}
%\studyprogramme{Informatika}
%\city{Košice} %mesto
\keywords{Programming, bioinformatics, data visualization, genome, covid-19}{Programovanie, bioinformatika, vizualizácia údajov, genóm, covid-19}
%\declaration{som nepodvadzal}

\abstract{%
    This bachelor thesis analyzes the general genome structure of different organisms (eukaryotes, prokaryotes and viruses) in order to understand the differences of various genomes and to come up with possible solutions for their visualization.
    Describes and compares some of popular existing programs that are designed for visualization of genome properties.
	As the next step, I perform 2D visualization and analysis of SARS-CoV-2 genome using some existing techniques and one developed by me.
    The results obtained during the visualization are verified and the code used to obtain them is composed into a stand-alone application.
}{%
    Bakalárska práca analyzuje všeobecnú štruktúru genómu rôznych organizmov (eukaryoty, prokaryoty a vírusy) s cieľom porozumieť rozdielom rôznych genómov a navrhnúť možné riešenia ich vizualizácie.
    Opisuje a porovnáva niektoré populárne existujúce programy, ktoré sú určené na vizualizáciu vlastnosti genómu.
    Zaobera sa 2D vizualizáciu a analýzu genómu SARS-CoV-2 nielen pomocou niektorých existujúcich, ale aj v práci vyvinutých techník.
    Výsledky dosiahnuté počas vizualizácie sa overuju a kód, použitý na ich získanie, prezentuje samostatnu aplikáciu.
}

\acknowledgment{Na tomto mieste by som rád poďakoval svojmu vedúcemu práce za jeho čas a odborné vedenie počas riešenia mojej záverečnej práce.

Rovnako by som sa rád poďakoval svojim rodičom a priateľom, najmä \textit{Adamovi Galuškovi} a \textit{Sultanu Shaimardanovi} za ich podporu a povzbudzovanie počas celého môjho štúdia.
    
V neposlednom rade by som sa rád poďakoval spoločnosti \textit{RedBull} a \textit{Ozzy Osbornovi} za energiu pri napísaní tejto práce.}

% if you want to work only on selected chapters
%\includeonly{chapters/analyza} %,chapters/synteza}

\thesisspec{chapters/ss.pdf}

% Load acronyms
\input{acronyms}


%% -----------------------------------------------------------------
%%          _                                       _   
%%       __| | ___   ___ _   _ _ __ ___   ___ _ __ | |_ 
%%      / _` |/ _ \ / __| | | | '_ ` _ \ / _ \ '_ \| __|
%%     | (_| | (_) | (__| |_| | | | | | |  __/ | | | |_ 
%%      \__,_|\___/ \___|\__,_|_| |_| |_|\___|_| |_|\__|
%%                                                      
%% -----------------------------------------------------------------

\begin{document}
%% Title page, abstract, declaration etc.:
\frontmatter{}

%% List of code listings, if you are using package minted
%\listoflistings

%\pagenumbering{arabic}

%% Chapters
% !TEX root = ../thesis.tex

\chaptermark{Úvod}
\phantomsection
\addcontentsline{toc}{chapter}{Úvod}

\chapter*{Úvod}

\par The order of DNA sequence and its variations are the very aspect which dictates the developmental processes of an organism, 
determines susceptibility to various diseases and uniquely identifies each creature. This area has always been on the periphery 
of the interests of scientific society, since the discovery in 1869 by Swiss-born biochemist Fredrich Miescher. 
For instance, The Human Genome Project (HGP) which started on October 1, 1990 and completed in April 2003 was one of the greatest feats 
of exploration in history of science. It was aimed at reading all the DNA sequences of our species, Homo sapiens. All in all, 
HGP introduced us the ability to read nature's complete genetic blueprint for building a human being. 
However, despite the successful completion of the project, a number of unknown DNA properties is still exists and demands the profound studying.

The most of the people do not know what bioinformatics is and this thesis is an attempt of diving in it.

The COVID-19 pandemic intoduced new challenges for the humanity, and bioinformatics particularly: after the sequencing every genome should be analyzed properly in order to obtain better understanding of its features.
Often, during the genome analysis different visualization techniques are used in order to portray data in an easy-to-understand manner.
Therefore, the visualization of SARS-CoV-2 genome structure and properties deserves special attention.

The first chapter of this thesis is aimed at analyzing the general genome structure of different organisms (eukaryotes, prokaryotes and viruses) in order to understand how to visualize it. 
Also, this chapter contains the overview of existing solutions for genome data representing.

The second chapter of this thesis contains the analysis of files that are used to store SARS-CoV-2 data, analysis and implementaton of different 2D visualiaton techniques to the coronavirus genome.
Moreover, this thesis suggests a new approach of how genome data can be visualized in order to see the smallest differences in genomes without profound investigation.
In addition, the second chapter also describes the composition of the software that is made of the scripts that were used to visualize SARS-CoV-2 genome.

Although several DNA processing tools exist, the problem of representing different genome properties which might vary at various species, 
concerning either the number of particular genes or complete chromosomes (if they are present), remains still actual. Moreover, 
the processing of the genome and its visualization demand an efficient approach, concerning the size of data and computational capabilities 
of an average computer. This work aims at representing some key genome properties in such a way.

\bigskip
\bigskip

{\noindent\LARGE{\textbf{Formulacia ulohy}}}

\smallskip
\smallskip

In this bachelor thesis I would like to analyze the genome structure of organisms, estimate the existing solutions to the genome data representing.
The very aim is to perform SARS-CoV-2 genome analysis, visualization and comprasion using different techniques and to compare the results with the existing solutions.
Another goal is to compose all the scripts that were used for the analysis and visualization into a stand-alone aplication.
% !TEX root = ../thesis.tex

\chapter{Analytická časť}

In most eukaryotic and prokaryotic organisms the hereditary material is either linear double-stranded DNA (deoxyribonucleic acid) molecules or a circular double-stranded DNA molecule. However, some extracellular life forms, might use RNA (ribonucleic acid) as the building block for their genome. For instance, viruses have a genome composed of either single-stranded DNA, double-stranded DNA or RNA, depending on the type of a virus. Therefore, a genome itself, is the complete content of genetic information in an organism, or in other words, all the unique DNA or RNA sequences the organism possesses. 
\section{Nucleotides}

Both of DNA and RNA  are polymeric molecules, that are composed of linear chains of various combinations of four different subunits, called nucleotides. The nucleotide itself is the basic unit of the DNA and RNA molecules, the monomer, which, however, could be found in the cell not only as the bearer of the genetic information, but also as a carrier of energy used to power enzymatic reactions. A five-carbon-atom sugar, a phosphate group and a nitrogenous base are three distinct components which, combined together, make up the quite complex nucleotide molecule. The combination of sugar and base is called a nucleoside, while the phosphate-sugar-base is termed a nucleotide. The nucleotide bases can be either a single-ringed pyrimidine or a double-ringed purine. Dinucleotide, trinucleotide and polynucleotide are the terms corresponding to two, three or many nucleotides connected with each other respectively.

\begin{figure}[!ht]
	\centering
	\includegraphics[width=.9\textwidth]{figures/bases}
	\caption{The structures of the pyrimidines and purines found in DNA and RNA. The sugar groups are highlighted in blue and the nitrogenous bases are highlighted in orange. The atoms of the sugar are numbered from 1 to 5. The atoms of the purine ring are numbered from 1 to 9, while those of the pyrimidine ring are numbered from 1 to 6. \label{o:latex_friendly_zone}}
\end{figure}

A nucleotide can be either a purine or pyrimidine. Guanine (G) and adenine (A)  are the common purines for both of DNA and RNA; the pyrimidine called cytosine (C) is also present in both nucleic acids. However, the pyrimidine uracil (U) is limited only to RNA, being replaced with thymine (T) in DNA. There are merely two base-pair combinations that are permissible – A base-paired with T (U) and C base-paired with G. It happens due to the geometries of the nucleotide bases and relative positions of atoms which participate in the connection. This property makes two sequences of polynucleotides in helix complement. Discrete nucleotides are attached to each other through sugar–phosphate bonds that connect the phosphate group on the 5’ carbon of one nucleotide with the hydroxyl group on the 3’ carbon of another nucleotide. The base pairing between adenine and thymine (uracil) involves two hydrogen bonds, while between cytosine and guanine involves three hydrogen bonds.
\section{Nucleodic acid spatial stucture}

As the three-dimensional structure of a nucleotide is not completely rigid, it is possible for DNA to have various spatial architectures: A-form, B-form, Z-form and the circular one. The position of the base relatively to the five-carbon-atom sugar can be changed by a rotation around the N-glycosidic bond and, in this way, significantly affect the three dimensional configuration of the molecule and helix consequently.

\begin{table}[!ht]
	\caption{DNA double helix}\label{t:1}
	\smallskip
	\centering
	
	\begin{tabular}{ |p{3cm}||p{3cm}|p{3cm}|p{3cm}|  }
		\hline
		\multicolumn{4}{|c|}{Features of the different conformations of the DNA double helix} \\
		\hline
		Feature& B-DNA & A-DNA & Z-DNA\\
		\hline
		\hline
		Type of helix & Right-handed & Right-handed & Left-handed\\
		\hline
		Number of base pairs per turn & 10 & 11 & 12\\
		\hline
		Distance between base pairs (nm) & 0.34 & 0.29 & 0.37\\
		\hline
		Distance per complete turn (nm) & 3.4 & 3.2 & 4.5\\
		\hline
		Diameter (nm) & 2.37 & 2.55 & 1.84\\
		\hline
		Major groove & Wide, deep & Narrow, deep & Flat\\
		\hline
		Minor groove & Narrow, shallow & Wide shallow & Narrow, deep\\
		\hline
	\end{tabular}
\end{table}

Moreover, although usually single-stranded, some RNA sequences have the ability to form a double helix. However, double helix RNA is rare and has nothing in common with the genome itself, since only the single-stranded RNA molecules appear to participate in some genome related processes in the eukaryotic and prokaryotic organisms. Since circular DNA may exist in several forms including single-stranded c-DNA, intact double-stranded c-DNA (closed circles with both strands covalently linked), nicked ds-c-DNA (only one strand covalently linked) and “concatenated circles” their properties are not described in the following table.

\section{Chromosomes in eukaryotic genomes}

In eukaryotic cells nucleic acid is situated in a membrane-bound organelle called the nucleus.\footnote{F RANCA , L. T. – C ARRILHO , E. – K IST , T. B. A review of DNA sequencing techniques.
Quarterly reviews of biophysics. 2002, 35, 02, s. 169–200}  The nuclear genome is split into a set of linear double-helix DNA molecules, each contained in a chromosome. No exceptions to this pattern are known: all eukaryotes that have been studied have at least two chromosomes and the DNA molecules are always linear. The only variability at this level of organization of eukaryotic genome is coherent with the number of chromosomes. Moreover, it appears, that biological features of an organism have no dependence on the number of chromosomes. 

\begin{figure}[!ht]
	\centering
	\includegraphics[width=.5\textwidth]{figures/nucleoDetailed}
	\caption{The nucleosome structure. H2A, H2B, H3 and H4 represent different types of histones. \label{o:latex_friendly_zone}}
\end{figure}

Despite the size of a nucleus (5-10 um), an overall length of DNA in the human cell is approximately 2.1m and can be packed inside the cell because of the method the nucleic acid is stored. The genetic material in viruses and bacteria consists of strings of DNA or RNA almost devoid of proteins. However, in eukaryotes, a substantial quantity of protein is associated with the DNA to form chromatin. At the lowest level, the DNA is organized by wrapping DNA strands around he proteins called histones, that contain a large amount of positively charged amino acids arginine and lysine. Those amino acids, and histones in general, play the crucial structural role, making it possible to bind the negative charged phosphate groups of the DNA nucleotides.

Averagely, the DNA rolled around the histones consists of 140-150 base pair, dependently on the species. Such a complex of DNA and histones is termed a nucleosome. These nucleosomes can be further coiled into increasingly larger coils up until forming chromosomes\footnote{C HAISSON , M. J. – P EVZNER , P. A. Short read fragment assembly of bacterial genomes.
Genome research. 2008, 18, 2, s. 324–330.}. However, tight coiling of DNA limits cells ability to access DNA and to process it.\footnote{Analysis of Genes
and Genomes Richard J. Reece University of Manchester, UK} Instead of being constantly coiled, the nucleic acid is usually found in a state called chromatin where some segments of acid are tightly reeled (heterochromatin), while other segments are entirely open (euchromatin). Euchromatin DNA is is highly accessible by the molecular complexes used by the cell and therefore is easier to manipulate with. 

The formation of nucleosomes represents the first level of packing, whereby
the DNA is reduced to about one-third of its original length. In the nucleus,
however, chromatin does not exist in this extended form.\footnote{Kipling, D. and Cooke, H.J. 1990. Hypervariable ultra-long telomeres in
mice. Nature 374: 400–402} Instead, the 10
nm chromatin fibre is further packed into a thicker 30 nm fibre, which was
originally called a solenoid. It is not clear whether the transition between the
10 nm fibre and the 30 nm fibre represents a physiological event or whether it
merely occurs in vitro as a consequence of altering the salt concentration. The
30 nm fibre does, however, consist of numerous nucleosomes packed closely
together, but the precise orientation and details of the structure are not clear.\footnote{Greider, C.W. 1996. Telomere length regulation. Annu. Rev. Biochem.
65: 337–365.}

The amount and extent of packing are determined by a sell, to control which sections of the genome can be expressed and which cannot. It affects cellular function and appears to be the predominant cause of differentiating cells type, while having the same DNA.

\begin{figure}[!ht]
	\centering
	\includegraphics[width=.9\textwidth]{figures/nucleosome1}
	\caption{Ncleosomes as the part of a chromosome.\label{o:latex_friendly_zone}}
\end{figure}

Transcription is the process by which an RNA copy of one of the strands in
the DNA double helix is made. The antisense strand of the DNA directs the
synthesis of a complementary RNA molecule. The RNA molecule produced is
therefore identical to the sense strand of the DNA – except that it contains U
instead of T. There are fundamental differences in the ways in which genes are
transcribed in prokaryotes and eukaryotes. Here, it is important to understand
the processes involved in each case. Many of the experiments we will look at
in later chapters involve the use of eukaryotic cells, but the bacterium E. coli
still plays a vital role in almost all genetic engineering experiments.
Transcription begins at specific DNA sequences called promoters. Like DNA
replication, transcription occurs in three phases – initiation, elongation and
termination. Initiation of transcription usually occurs to the 3 side of the
promoter, and termination occurs at specific sites downstream of the coding
sequence of the gene. At first glance, the overall architecture of a typical
prokaryotic gene and a typical eukaryotic gene may appear to be similar. However, the controlling region for eukaryotic genes will not
function in a prokaryotic cell, and vice versa.\footnote{Levis, R.W. 1989. Viable deletions of a telomere from a Drosophila
chromosome. Cell 58: 791–801.}
Most protein coding genes in prokaryotes are transcriptionally active by
default. That is to say, in the absence of other factors, the RNA polymerase can
recognize the promoter of a gene, bind to it and produce RNA. Transcriptional
control is brought to bear on the gene by repressor proteins that bind to DNA
sequences adjacent to the RNA polymerase binding site. DNA binding by the
repressor either occludes RNA polymerase binding and/or prevents a bound
polymerase from transcribing. The eukaryotic RNA polymerase involved in the
production of protein coding genes is unable to recognize promoter
sequences on its own. Therefore, eukaryotic genes are transcriptionally inactive
in the absence of other factors. In both prokaryotes and eukaryotes, transcrip-
tion is a highly regulated process. Proper timing and levels of gene expression
are essential to almost all cellular processes.
% !TEX root = ../thesis.tex

\chapter{Syntetická časť}
\label{methodology}
% !TEX root = ../thesis.tex

\chapter{Vyhodnotenie}

During the first part of this thesis I have analyzed the general genome structure of different organisms, DNA and RNA from the molecular, biological and informatical point of view.

The spatial structure of DNA and its differences were listed at table 1.1 .

The genomes of eukaryotes, prokaryotes and viruses were described in corresponding sections.
The description of those genomes includes their unique properties, such as intron-exon structures mainly present in genomes of eukaryotes and gene clusters present in genomes of prokaryotes.
The genes, their types, locations, functions and patterns that allow to find them (ORFs) were analyzed and properly described.

The next step was to classify existing solutions for genome data representation in order to achieve the understanding how the modern software works and what it can suggest to users worldwise.
They can be classifed as \textit{web-based} and \textit{stand-alone applications}.

Moreover, two categories of existing genome browsers were precisely described:
\begin{itemize}
    \item \textbf{Species-independent} solutions that are capable of visualizing any genome, including Ensemble genome browser, UCSC genome browser and GBrowse framework were analyzed and compared (table 1.2).
    \item \textbf{Species-specific} solutions that are aimed at visualization of particular species, including MSU rise genome browser and Rice-Map genome browser.
\end{itemize}

After, due to the extreme complexity of prokaryotic and eukaryotic genomes and due to the world pandemic, the SARS-CoV-2 virus genome was chosen to be visualized.

During the second part of this thesis, in order to understand how and where genome data is being stored, I have performed a comprehensive analysis of formats that are usually used to store genome-coherent information: FASTA file format, GFF and GenBank (GBK) file formats.
The SARS-CoV-2 related files were used as an example for the very analysis.

I came to a conclusion, that genome visualization can be done in two ways:
\begin{itemize}
    \item The first way is to visualize raw data which is a nucleotide sequence for DNA and RNA or an amino acid sequence (using information in FASTA files).
    \item The second way is to visualize previously processed and well studied data, that contain gene locations, ORFs positions, etc. and which is stored in the genome annotation files (GFF and GenBank files).
\end{itemize}

After that, I have performed the analyzis and visualization of SARS-CoV-2 genome using the following methods.

\textbf{Nucleotides distribution and GC-content} analysis of FASTA file contents have shown, that SARS-CoV-2 genome is composed of 29903 nucleoties (basic units of genome).
The distribution of them was the following: adenine (A) appeared at the genome sequence 8954 times, thymine (T) - 9594, cytosine (C) - 5492 and  guanine (G) - 5863 times.
By knowing this data, I have computed the GC-content property of a genome that was low and appeared to be 37.97\%.

This property is important, because it shows that SARS-CoV-2 genome has small number of ORFs (genes), because they are usually begin at GC-rich regions of genome.

\textbf{Gates's method} visualization was chosen to be performed, since I have accidently rediscovered it while thinking of how genome can possibly be visualized.
It also processes the FASTA file.
After performing this visualization, I have noticed that strange sequence of 33 adenine (A) nucleotides ends SARS-CoV-2 genome (figure 2.4).
To make sure, that no errors were made during the visualization I have checked the very FASTA sequence and came up with the conclusion that the visualization was done in the right way.
    
Despite the degeneracy, which is the main disadvantage of the method, it can plot DNA patterns without machine examination, as I have shown by visualizing the first chromosome of the smallest known eukaryotic genome that belongs to \textit{Encephalitozoon Intestinalis} (figure 2.6).

To understand whether this method is suitable for visual comprasion of genomes of related species, I have visualized SARS-CoV-2 genome and genome of its closest relative SARS-CoV-1 virus next to each other.
I have made a hypothesis that they must be very similar since they resemble each other in the terms of achieved visualization (figure 2.7), and to prove it, I have run the pairwise2 algorithm that confirmed, that their percentage of similarity is equal to 83.34\%.
I have chosen that algorithm since it is among the best alignment algorithms.

\textbf{2D Matrix method} visualization was chosen to be performed, since tandem repeats in a genome sequence might be visually distinguishable without any machine examination.
However, after the very visualization, I did not manage to find any of them (figure 2.9).
This method also requires FASTA file with sequence.

The main disadvantage of this method is that point mutations in genome, or even significant ones, are almost not noticabe without comprehensive analyzis of the obtained image.

\textbf{2D Matrix method improvement} was intoduced by me to cope with the disadvantage of the previous method by using a hash-function.

I proved it by comparing original genome of SARS-CoV-2 and the same genome with a point mutation at the second nucleotide (T was substituted with G) using my method (figure 2.10).

\textbf{Amino acid retrieval} was performed to obtain SARS-CoV-2 proteins for required for the next method. 

After achievment of proteins from the DNA sequence stored in a FASTA file, I have compared the results with existing SARS-CoV-2 proteins by performing a BLAST search.
The compasion have shown, that almost no mistakes were made since the similarity ratio was very high.
Because of my interest, I have also compared them with proteins of relative species that have also shown a high ratio of similarity (table 2.2).

\textbf{ORF identification and visualization} was performed using the GenBank annotation file in order to visualize those parts of SARS-CoV-2 genome, that are probably genes.

After searching for ORFs locations in the GBK file and comparing their indexes with the proteins obtained previously, removing the short ones, I have achieved 6 coding sequenses of genome (genes).
The visualiation of achieved genes among other ORFs is visible at the figure 2.11 .

To verify the results, BLAST search was performed (table 2.3).
It has shown, that I have visulized those genes properly, since they have a high similarity with the existing ones.

\smallskip
The next and the last step was to combine all the code that was used into a simple stand-alone console application which is capable of visualizing SARS-CoV-2 genome using previously described techniques.
The detailed software architecture is described at the corresponding section.

From the main disadvantages of the developed program I can admit that it is rather simlple, has no graphical interface and, at the moment, not all methods support visualiaton of any genome.
Therefore, the developed visualiation tool can be claffied as \textit{species-specific}.

\label{evaluation}
% !TEX root = ../thesis.tex

\chapter{Záver}
\label{summary}

Počas práce na tejto bakalárskej práci som sa ponoril do oblasti bioinformatiky, komplexne som analyzoval štruktúru genómu rôznych organizmov, porovnával existujúce riešenia na vizualizáciu údajov a štruktúry genomu a vyvíjal nové.

Vyvinutý program pracuje so súbormi fórmatu FASTA a GenBank a je schopný vizualizovať genóm SARS-CoV-2 a genomy inych organizmov pomocou rôznych 2D vizualizačných techník a poskytovať základe štatistické údaje o genóme.
Taktiež program využivá na vizualizáciu vylepšenú 2D maticovú metódu, ktorá umôžňuje identifikovať najmenšie rozdiely v genómoch a bola vyvinutá v rámci predloženej práce.

Budúce vylepšenia vyvinutéj aplikácie sa môžu zamerať hlavne na vylepšenie existujúcich vizualizačných techník, využitie ďalších, vizualizáciu d'alšich vlastností genomov, pripadne pridanie podpory pre väčšie genómy, a na pridanie grafického rozhranía ku programu.

% good linebraking of bibtex url
\setcounter{biburllcpenalty}{7000}
\setcounter{biburlucpenalty}{8000}

% %% The bibliography
\printbibliography[heading=bibintoc]

% \label{theend} % the last page of the thesis

% % List of acronyms
% \printglossary[type=\acronymtype,title={\acrlistname}]

% % Glossaries
% \printglossary

%% Appendix
% !TEX root = ../thesis.tex

\chapter*{Zoznam príloh}
\addcontentsline{toc}{chapter}{Zoznam príloh}

\begin{description}
    \item[Príloha A] Dokumentácia ku programu 
    \item[Príloha B] Vyvinutý program
    \item[Príloha C] CD médium -- záverečná práca v~elektronickéj podobe
\end{description}

\appendix
\renewcommand\chaptername{Príloha A}
% !TEX root = ../thesis.tex


\thispagestyle{empty}
	\begin{center}
		\vspace*{1cm}
		
		\textbf{\large Technicka Univerzita v Košiciach }
		
		\vspace{0.4cm}
		Fakulta elektrotechniky a informatiky
		
		\vspace{4.5cm}
		
		\textbf{\Large Dokumentácia k programu na vizualizáciu štruktúry genómu\\}
		\vspace{1.5cm}
		Príloha A k bakalárskej práci
		\vfill
		
		
		\vspace{2.8cm}
		
		{\raggedleft\vfill{%
		 		
			}\par}
		
		{\raggedright\vfill{%
				2021 \quad\quad\quad\quad\quad\quad\quad\quad\quad\quad\quad\quad\quad\quad\quad\quad\quad\quad\quad\quad\quad\quad\quad Oleksandr Korotetskyi
			}\par}
		
		
		
	\end{center}

\newpage
\rhead{Dokumentácia}
\addcontentsline{toc}{chapter}{Documentácia}
\subsubsection{\Large{Úvod}}
\addcontentsline{toc}{section}{Úvod}
Túto dokumentáciu možno považovať za používateľskú príručku k použitiu programu a ako systémovú príručku, ktorá popisuje funkčnosť a architektúru programu.

Dokumentácia je rozdelená na niekoľko častí, ktoré sa venujú zodpovedajúcim témam: inštalácia, scenáre spustenia a použitia, architektúra aplikácie.

\subsubsection{\Large{Inštalácia}}
\addcontentsline{toc}{section}{Inštalácia}
Aplikácia bola vyvinutá pre použitie hlavne na platformách Unix / Linux, a preto môže pokus o jej inštaláciu na platformu Windows viesť k neočakávanému správaniu programu.


\subsubsection{Python 3.8}
Aplikácia je napísaná v Pythone 3.8, a preto on je nevyhnutný pre použitie programu.
V systémoch založených na Debiane je možné ho nainštalovať pomocou nasledujúcich príkazov v termináli (štandardnóm príkazovóm riadku):
\begin{lstlisting}[language=bash]
  $ sudo apt-get update
  $ sudo apt-get install python3.8
\end{lstlisting}

Pre systémy založené na Fedore by sa mal použiť nasledujúci príkaz:
\begin{lstlisting}[language=bash]
  $ sudo dnf install python3
\end{lstlisting}


\subsubsection{Pip}
Ďalším krokom je inštalácia {\fontfamily{lmtt}\selectfont pip} (správca balíkov python), ktorý sa použije na inštaláciu ďalších závislostí.
Je to môžne urobiť pomocou nasledujúceho príkazu pre distribúcie založené na Debiane:
\begin{lstlisting}[language=bash]
  $ sudo apt-get install python3-pip
\end{lstlisting}

A pre systémy založené na Fedore musia byť použité:
\begin{lstlisting}[language=bash]
  $ curl "https://bootstrap.pypa.io/get-pip.py" -o "get-pip.py"
  $ python get-pip.py
\end{lstlisting}


\subsubsection{Knižnice}
Softvér pracuje na populárnych knižniciach, ktoré poskytujú používateľovi rozšírené funkcie pre sprácovanie genomov a ine účely.
Aby bolo možné aplikáciu používať, je potrebné nainštalovať nasledujúcich 12 balíkov:
{\fontfamily{lmtt}\selectfont
\begin{itemize}
    \item bio==0.4.1
    \item biopython==1.78
    \item matplotlib==3.4.2
    \item numpy==1.19.5
    \item pandas==1.2.4
    \item pillow==8.2.0
    \item pyparsing==2.4.7
    \item requests==2.25.1
    \item seaborn==0.11.1
    \item urllib3==1.26.4
    \item bcbio-gff
    \item dna\_features\_viewer
\end{itemize}
}

Úplný zoznam požadovaných balíkov sa nachádza v súbore \textbf{\fontfamily{lmtt}\selectfont requirements.txt}, ktorý je umiestnený v koreňovom adresári programu.

Samotnú inštaláciu knižníc je možné vykonať v termináli v koreňovom adresári programu pomocou jedného z nasledujúcich príkazov, ktoré sa môžu líšiť v závislosti od systému:
\begin{lstlisting}[language=bash]
  $ pip install -r requirements.txt
\end{lstlisting}
\begin{lstlisting}[language=bash]
  $ pip3 install -r requirements.txt
\end{lstlisting}

\addcontentsline{toc}{section}{Popis spustenia a činnosti aplikácie}
\subsubsection{\Large{Popis spustenia a  činnosti aplikácie}}
Tento vizualizačný nástroj podporuje dva režimy vykonávania: \textit {verbose} a \textit {quiet}.
Aplikácia v režime \textit{verbose} poskytuje používateľovi komentáre a prostredie na triviálnu interakciu, zatiaľ čo režim \textit{quit} je vhodnejší na účely rýchlejšej vizualizácie a automatizácie.

Oba režimy majú rovnaké funkcie, a preto sa líšia iba v tom, ako používateľ povie programovi, čo má robiť.
V režime \textit{verbose} používateľ komunikuje s programom prostredníctvom vstupu a výstupu konzoly, zatiaľ čo v režime \textit{quit} používa iba argumenty príkazového riadku.

Táto časť popisuje rôzne scenáre vykonávania programu v režime \textit{verbose} a sprevádzané príkazom na vykonanie rovnakej akcie iba pomocou argumentov príkazového riadku v režime \textit{quiet}.

Na spustenie aplikácie v režime \textit{verbose} by sa mal použiť jeden z nasledujúcich príkazov:
\begin{lstlisting}[language=bash]
  $ python3 Main.py
\end{lstlisting}
\begin{lstlisting}[language=bash]
  $ python3 Main.py -m v
\end{lstlisting}

Po spustení aplikácie sa objavi hlavné menu nástroja:
\begin{lstlisting}[language=bash]
  +----------------------------------+
  |--- Welcome to the Visualizer! ---|
  +----------------------------------+
  Choose the option: 
  1. Download SARS-CoV-2 genome sequence & associated files
  2. Plot sequence statistics
  3. Gates' visualization
  4. 2D Matrix visualization
  5. Improved 2D Matrix visualization
  6. Plot ORFs
  7. Compare genomes
  8. Exit
  Choice: 
\end{lstlisting}
Užívateľ je schopný voliť rôzné možnosti zadaním zodpovedajúceho im čísla.
Na ukončenie práci s nástrojem je potrebné stlačiť zadať „q“ pri hoci akom vstupe, na ktorý program čaká.


\subsubsection{Scenár 1}
Po výbere prvej možnosti sa v prípade úspechu zobrazia nasledujúce správy.
\begin{lstlisting}[language=bash]
  Necessary files are being downloaded...
  Done!
\end{lstlisting}
Všetky potrebné súbory (SARS-CoV-2.fasta a SARS-CoV-2.gb) sa úspešne stiahli.
Všetky ostatné súbory na vizualizáciu genómom musí používateľ pridať ručne do adresára {\fontfamily{lmtt}\selectfont data}.

\bigskip
Rovnaký scenár je možné vykonať v režime \textit{quiet} bez akýchkoľvek programových správ pomocou nasledujúceho príkazu:
\begin{lstlisting}[language=bash]
  $ python3 Main.py -m q -d
\end{lstlisting}


\subsubsection{Scenár 2}
Po výbere druhej možnosti program požiada používateľa, aby si vybral postupnosť z tých, ktoré sa nachádzajú v adresári {\fontfamily{lmtt}\selectfont data}.
\begin{lstlisting}[language=bash]
  Choose the sequence to plot the statistics of:
  1. alteromonas.fasta
  2. SARS-CoV-2.fasta
  3. ebola.fasta
  Choice: 2
\end{lstlisting}
Ďalším krokom je určenie intervalu, o ktorom sa musia štatistické údaje zobrazovať:
\begin{lstlisting}[language=bash]
  Specify the interval (0 for the entire genome)
  Start:	1223
  End:	0
\end{lstlisting}

Nuly predstavujú defaultné hodnoty, zatiaľ čo program vytvára tento výstup:
\begin{lstlisting}[language=bash]
  Frequencies of nucleotides on the interval [1223;29903]:
  A:	8626
  T:	9251
  G:	5577
  C:	5226
  Total:	28680
  GC-content on interval [1223;29903]:	0.3767%
\end{lstlisting}
Zobrazuju sa základné štatistické údaje analyzovanej sekvencie, ktoré by mohli byť užitočné.


\bigskip
Rovnaký scenár je možné vykonať v režime \textit{quiet} pomocou nasledujúceho príkazu:
\begin{lstlisting}[language=bash]
  $ python3 Main.py -m q -s --pos 1223 0
\end{lstlisting}



\subsubsection{Scenár 3}
Po výbere tretej možnosti program požiada používateľa, aby si vybral postupnosť z tých, ktoré sa nachádzajú v adresári {\fontfamily{lmtt}\selectfont data}.
\begin{lstlisting}[language=bash]
  Choose the sequence to visualize using Gates' method:
  1. alteromonas.fasta
  2. SARS-CoV-2.fasta
  3. ebola.fasta
  Choice: 3
\end{lstlisting}
Ďalším krokom je určenie intervalu sekvencie genómu, ktorý sa bude vizualizovať:
\begin{lstlisting}[language=bash]
  Specify the interval (0 for the entire genome)
  Start:	0
  End:	8000
  Done!
\end{lstlisting}

Generovaný obrázok je uložený v adresári {\fontfamily{lmtt}\selectfont out}.

\bigskip
Rovnaký scenár je možné vykonať v režime \textit{quiet} pomocou nasledujúceho príkazu:
\begin{lstlisting}[language=bash]
  $ python3 Main.py -m q -g ebola.fasta --pos 0 8000
\end{lstlisting}

\subsubsection{Scenár 4}
Po výbere štvrtej možnosti program požiada používateľa, aby si vybral postupnosť z tých, ktoré sa nachádzajú v adresári {\fontfamily{lmtt}\selectfont data}.
\begin{lstlisting}[language=bash]
  Choose the sequence to visualize using 2D Matrix method:
  1. alteromonas.fasta
  2. SARS-CoV-2.fasta
  3. ebola.fasta
  Choice: 3
  Done
\end{lstlisting}

Táto vizualizácia nepodporuje voľbu intervalu a vygenerovaný obrázok sa ukláda do adresára {\fontfamily{lmtt}\selectfont out}.

\bigskip
Rovnaký scenár je možné vykonať v režime \textit{quiet} pomocou nasledujúceho príkazu:
\begin{lstlisting}[language=bash]
  $ python3 Main.py -m q -x ebola.fasta
\end{lstlisting}

\subsubsection{Scenár 5}
Po výbere piatej možnosti program požiada používateľa, aby si vybral postupnosť z tých, ktoré sa nachádzajú v adresári {\fontfamily{lmtt}\selectfont data}.
\begin{lstlisting}[language=bash]
  Choose the sequence to visualize using 2D HMatrix method:
  1. alteromonas.fasta
  2. SARS-CoV-2.fasta
  3. ebola.fasta
  Choice: 2
\end{lstlisting}

Ďalším krokom je zadavánie veľkosti vytvoreného obrázka a po ňom program vygeneruje nasledujúci výstup:
\begin{lstlisting}[language=bash]
  Specify the size of image (preferably a power of 2, >= 512):
  Size (px):	2048
  seed: f9fa11164acc370f5c187a286c25dcffe0b93363c68ce5d658d83e
  w, h: 1024.0, 512.0
  w, h: 256.0, 256.0
  opacity: 50%
  w, h: 128.0, 64.0
  opacity: 37%
  w, h: 32.0, 32.0
  opacity: 25%
  w, h: 16.0, 8.0
  opacity: 15%
  w, h: 4.0, 4.0
  opacity: 9%
  w, h: 2.0, 1.0
  opacity: 5%
  Done
\end{lstlisting}

Vygenerovaný obrázok sa ukláda do adresára {\fontfamily{lmtt}\selectfont out}.

\bigskip
Rovnaký scenár je možné vykonať v režime \textit{quiet} bez výstupu konzoly pomocou nasledujúceho príkazu:
\begin{lstlisting}[language=bash]
  $ python3 Main.py -m q -i ebola.fasta -S 2048
\end{lstlisting}



\subsubsection{Scenár 6}
Po výbere šiestej možnosti program požiada používateľa, aby si vybral anotačný súbor genómu, ktorý sa má použiť, z tých, ktoré sa nachádzajú v adresári {\fontfamily{lmtt}\selectfont data}.
\begin{lstlisting}[language=bash]
  Choose the annotation file to visualize ORFs:
  1. ebola.gb
  2. SARS-CoV-2.gbk
  Choice: 2
  Done  
\end{lstlisting}
Vygenerovaný obrázok sa úspešne ulkláda do adresára {\fontfamily{lmtt}\selectfont out}.

\bigskip
Rovnaký scenár je možné vykonať v režime \textit{quiet} bez výstupu konzoly pomocou nasledujúceho príkazu:
\begin{lstlisting}[language=bash]
  $ python3 Main.py -m q -o SARS-CoV-2.gbk
\end{lstlisting}



\subsubsection{Scenár 7}
Po výbere siedmej možnosti program požiada používateľa, aby vybral sekvencie genómu na porovnanie so sekvenciami, ktoré sa nachádzajú v adresári {\fontfamily{lmtt}\selectfont data}.
\begin{lstlisting}[language=bash]
  Choose the first sequence to compare:
  1. alteromonas.fasta
  2. SARS-CoV-2.fasta
  3. ebola.fasta
  Choice: 2
  Choose the second sequence to compare:
  1. alteromonas.fasta
  2. SARS-CoV-2.fasta
  3. ebola.fasta
  Choice: 3
  Similarity (%): 72  
\end{lstlisting}
Po samotnom porovnaní program zadá percento podobnosti.

\bigskip
Rovnaký scenár je možné vykonať v režime \textit{quiet} bez výstupu konzoly pomocou nasledujúceho príkazu:
\begin{lstlisting}[language=bash]
  $ python3 Main.py -m q -c SARS-CoV-2.fasta ebola.fasta
\end{lstlisting}


\subsubsection{Použitie režimu Quiet}
Pre ziskanie dokladnéj informácií o režime \textit{quiet}, používateľ môže zadať nasledujúci príkaz:
\begin{lstlisting}[language=bash]
  $ python3 Main.py -h
\end{lstlisting}

Tento príkaz zobrazuje všetky podporované argumenty príkazového riadku:
\begin{lstlisting}[language=bash]
usage: Main.py [-h] [-m {q,v}] [-d] [-g GATES] [-o ORF] [-s] 
               [-x MATRIX] [-i HASH] [-c COMP COMP] [-S SIZE]
               [-p POS POS] [-a] [-n NAME]

optional arguments:
  -h, --help            show this help message and exit
  -m {q,v}, --mode {q,v}
                        Execution mode: quiet / verbose
  -d, --download        Download SARS-CoV-2 genome associated 
                        files
  -g GATES, --gates GATES
                        Perform Gates' visualization. Parameter
                        is an input sequence filename
  -o ORF, --orf ORF     Plot ORFs of the genome. Parameter is 
                        an input sequence filename
  -s, --stat            Obtain genome statistical data including
                        the distribution of nucleotides 
                        and a GC-content
  -x MATRIX, --matrix MATRIX
                        Plot the nucleotide sequence into a 2D 
                        matrix. Parameter is the input sequence 
                        filename
  -i HASH, --hmatrix HASH
                        Plot the nucleotide sequense into Hashed
                        2D matrix. Parameter is input sequence 
                        filename
  -c COMP COMP, --compare COMP COMP
                        Compare specified genome sequences using 
                        the pairwise2 algorithm
  -S SIZE, --size SIZE  Size of the picture side in pixels
  -p POS POS, --pos POS POS
                        Start and end positions of the nucleotide 
                        sequence to perform an action (0 for default)
  -a, --all             Perform all possible actions but comparison
                        in the default mode
  -n NAME, --name NAME  Input sequnce filename
\end{lstlisting}


\rhead{Dokumentácia}
\subsubsection{\Large{Architektúra aplikácie}}
\addcontentsline{toc}{section}{Architektúra aplikácie}
Vyvinutý program predstavuje samostatnú konzolovú aplikáciu, ktorá sa skladá z 8 modulov umiestnených v koreňovom adresári programu.
Obsah adresára je uvedený nižšie:
\textbf{\fontfamily{lmtt}\selectfont
\begin{itemize}
    \item Main.py
    \item Comparison.py
    \item GatesVisualization.py
    \item MatrixVisualization.py
    \item HMatrixVisualization.py
    \item ORFPlotter.py
    \item SeqCollector.py
    \item StatGenerator.py
    \item requirements.txt
\end{itemize}
}

Počas chodu programu sa vytvárajú dva ďalšie adresáre so súbormi, ak neexistujú: \textbf{\fontfamily{lmtt}\selectfont data} a \textbf{\fontfamily{lmtt}\selectfont out}.

Prvý adresár obsahuje stiahnuté sekvencie a program ho považuje za zdrojový adresár všetkých sekvencií genómu a súborov anotácií genómu, s ktorými program pracuje.
Preto, aby bolo možné vizualizovať a pracovať s vlastnými genómami, ich súbory musia byť vložené do adresára \textbf{\fontfamily{lmtt}\selectfont data}.

Druhý slúži na uloženie všetkých výstupných obrázkov formátu {\fontfamily{lmtt}\selectfont .png}, ktoré program vytvorí.
Preto, aby si užívateľ mohol pozrieť vykonané vizualizácie, musí ich vyhľadať v adresári \textbf{\fontfamily{lmtt}\selectfont out}.

\rhead{Dokumentácia}
\subsubsection{Modules description}
\textbf{\fontfamily{lmtt}\selectfont Main.py} je hlavný modul programu, ktorý je zodpovedný za použitie zvyšných modulov na vykonanie zadanej úlohy.
Zaoberá sa vstupom a výstupom z konzoly, navrhuje dostupné metódy vizualizácie a získava podrobnosti potrebné na ich výkon.
Obsahuje nasledujúce funkcie:
\begin{itemize}
  \item \textbf{\fontfamily{lmtt}\selectfont verifyArgs()} -- overuje a kontroluje argumenty príkazového riadku, ak je program spustený v režime \textit{quiet}. Ak sa vyskytne chyba, program sa zastaví.
  \item \textbf{\fontfamily{lmtt}\selectfont welcomeBanner()} -- ak je zapnutý režim \textit{verbose}, zobrazí "uvítací banner".
  \item \textbf{\fontfamily{lmtt}\selectfont mainMenu()} -- ak je režim \textit{verbose} zapnutý, zobrazí hlavné menu aplikácie a požiada používateľa, aby vybral možnosť pokračovania; skontroluje vstup používateľa. Vráti číslo vybratého scenára.
  \item \textbf{\fontfamily{lmtt}\selectfont vObtainFiles(msg)} -- ak je zapnutý režim \textit{verbose}, vypíše všetky súbory vo formáte {\fontfamily{lmtt}\selectfont FASTA} a požiada používateľa, aby si jeden vybral. Parameter {\fontfamily{lmtt}\selectfont msg} predstavuje správu, ktorá sa má zobraziť. Vráti názov vybraného súboru, ak je prítomný, v opačnom prípade sa zobrazí chybové hlásenie a funkcia vráti hodnotu {\fontfamily{lmtt}\selectfont None}.
  \item \textbf{\fontfamily{lmtt}\selectfont vObtainFiles2(msg)} -- ak je zapnutý režim \textit{verbose}, vypíše všetky súbory vo formáte {\fontfamily{lmtt}\selectfont GenBank} a požiada používateľa, aby si jeden vybral. Parameter {\fontfamily{lmtt}\selectfont msg} predstavuje správu, ktorá sa má zobraziť. Vráti názov vybraného súboru, ak je prítomný, v opačnom prípade sa zobrazí chybové hlásenie a funkcia vráti hodnotu {\fontfamily{lmtt}\selectfont None}.
  \item \textbf{\fontfamily{lmtt}\selectfont vObtainInterval()} -- ak je režim \textit{verbose} zapnutý, požiada používateľa, aby určil interval sekvencie genómu, na ktorom má vykonať akciu. Vráti pozície {\fontfamily{lmtt}\selectfont start} a {\fontfamily{lmtt}\selectfont end} po ich overení.
  \item \textbf{\fontfamily{lmtt}\selectfont vObtainSize()} -- ak je zapnutý režim \textit{verbose}, požiada používateľa, aby určil veľkosť pre vygenerovanie štvorcového obrázku. Vráti veľkosť strany obrázka v pixeloch.
  \item \textbf{\fontfamily{lmtt}\selectfont main()} -- vykoná hlavný cyclus programu a určuje, ktorú akciu má vykonať podľa argumentov príkazového riadku a vstupu používateľa. V režime \textit{quiet} končí program po vykonaní akcie, zatiaľ čo v režime \textit{verbose} znova zobrazí hlavné menu aplikácie.
\end{itemize}



\textbf{\fontfamily{lmtt}\selectfont SeqCollector.py} je zodpovedný za stiahnutie všetkých požadovaných sekvencií a súborov anotácií z databázy NCBI pre vizualizáciu genómu SARS-CoV-2.
V tejto chvíli nepodporuje sťahovanie súborov spojených s inými genómami.
Obsahuje nasledujúce funkcie:
\begin{itemize}
  \item \textbf{\fontfamily{lmtt}\selectfont downloadFiles()} -- vytvorí adresár {\fontfamily{lmtt}\selectfont data}, ak neexistuje, a stiahne (vyžaduje sa internetové pripojenie) sekvenciu genómu a anotačné súbory SARS-CoV-2.
\end{itemize}



\textbf{\fontfamily{lmtt}\selectfont StatGenerator.py} získava štatistické údaje, ako je obsah GC a distribúcia nukleotidov / aminokyselín.
Užívateľ si môže zvoliť oblasť genómu, ktorá sa má štatisticky analyzovať.
Obsahuje nasledujúce funkcie:
\begin{itemize}
  \item \textbf{\fontfamily{lmtt}\selectfont getStats(filename, mode, start, end)} -- overuje typ {\fontfamily{lmtt}\selectfont filename}. Overuje {\fontfamily{lmtt}\selectfont start} a {\fontfamily{lmtt}\selectfont end} pozície. Funkcia zastaví vykonávanie programu v chybových prípadoch a zobrazí príslušné chybové hlásenie. Vypisuje štatistické údaje o zadanom intervale sekvencie genómu a poskytne používateľovi ďalšie komentáre v režime \textit{verbose}.
\end{itemize}



\textbf{\fontfamily{lmtt}\selectfont GatesVisualization.py} vykonáva vizualizáciu pomocou Gatesovej metódy do súboru {\fontfamily{lmtt}\selectfont -Gates.png} .
Užívateľ je schopný zvoliť oblasť genómu ktorú chce vizualizovať.
Obsahuje nasledujúce funkcie:
\begin{itemize}
  \item \textbf{\fontfamily{lmtt}\selectfont visualize(filename, mode, start, end)} -- overuje typ {\fontfamily{lmtt}\selectfont filename}. Overuje {\fontfamily{lmtt}\selectfont start} a {\fontfamily{lmtt}\selectfont end} pozície. Funkcia zastaví vykonávanie programu v chybových prípadoch a zobrazí príslušné chybové hlásenie. Vykonáva Gatesovu vizualizáciu určeného intervalu sekvencie genómu.
  \item \textbf{\fontfamily{lmtt}\selectfont save(outFileName, image)} -- vytvorí adresár {\fontfamily{lmtt}\selectfont out} ak neexistuje, a uloží vygenerovaný obrázok {\fontfamily{lmtt}\selectfont outFileName} do adresára {\fontfamily{lmtt}\selectfont out}.
\end{itemize}



\textbf{\fontfamily{lmtt}\selectfont MatrixVisualization.py} kreslí zvolený genóm pomocou generácie 2D matice do súboru {\fontfamily{lmtt}\selectfont -Matrix.png} .
Veľkosť výstupného obrázka sa počíta automaticky.
Obsahuje nasledujúce funkcie:
\begin{itemize}
  \item \textbf{\fontfamily{lmtt}\selectfont visualize(filename)} -- overuje typ {\fontfamily{lmtt}\selectfont filename}. Funkcia zastaví vykonávanie programu v chybových prípadoch a zobrazí príslušné chybové hlásenie. Vykoná vizualizáciu sekvencie špecifikovaného genómu jeho vykreslením do 2D matice.
\end{itemize}



\textbf{\fontfamily{lmtt}\selectfont HMatrixVisualization.py} kreslí genóm do 2D matice vybranej veľkosti pomocou algoritmu hash funkcie do súboru {\fontfamily{lmtt}\selectfont -Hmatrix.png} .
Veľkosť výstupného obrázka môže byť zádana používateľom.
Obsahuje nasledujúce funkcie:
\begin{itemize}
  \item \textbf{\fontfamily{lmtt}\selectfont save(outFileName, image)} -- vytvorí adresár {\fontfamily{lmtt}\selectfont out}, ak neexistuje, a uloží vygenerovaný obrázok{\fontfamily{lmtt}\selectfont outFileName} do adresára {\fontfamily{lmtt}\selectfont out}.
  \item \textbf{\fontfamily{lmtt}\selectfont drawLayer(imgSize, depth, mode)} -- kreslí farébne bloky na základe {\fontfamily{lmtt}\selectfont getRandomColor()} vo veľkosti {\fontfamily{lmtt}\selectfont getBlockSize(imgSize, depth)} pre jednotlivé vrstvy. Vráti obrázok aktuálnej vrstvy. Poskytuje používateľovi ďalšie komentáre v režime \textit{verbose}.
  \item \textbf{\fontfamily{lmtt}\selectfont getHash(filename)} -- počíta hash sekvencie genómu {\fontfamily{lmtt}\selectfont filename}.
  \item \textbf{\fontfamily{lmtt}\selectfont getRandomColor()} -- vráti n-ticu náhodných hodnôt farieb vo formáte RGB.
  \item \textbf{\fontfamily{lmtt}\selectfont getBlockSize(imgSize, depth)} -- počíta a vracia {\fontfamily{lmtt}\selectfont width} a {\fontfamily{lmtt}\selectfont height} podľa {\fontfamily{lmtt}\selectfont imgSize} a {\fontfamily{lmtt}\selectfont depth}. S každou iteráciou cyklu sa každá strana bloku delí na polovicu alebo na štvrtiny, v závislosti od {\fontfamily{lmtt}\selectfont depth}.
  \item \textbf{\fontfamily{lmtt}\selectfont visualize(filename, mode, ssize)} -- skontroluje typ {\fontfamily{lmtt}\selectfont filename} a zastaví vykonávanie programu v prípade chýb. Vykonáva 2D Hashed Matrix vizualizáciu určenej sekvencie genómu rekurzívnym spôsobom. Upraví {\fontfamily{lmtt}\selectfont seed} na generovanie náhodných čísel na základe funkcie {\fontfamily{lmtt}\selectfont getHash(filename)}. Zlúči vrstvy veľkosti {\fontfamily{lmtt}\selectfont size} vytvorené funkciou {\fontfamily{lmtt}\selectfont drawLayer(size, depth, mode)} v závislosti od definovanej {\fontfamily{lmtt}\selectfont opacity}. Poskytuje používateľovi ďalšie komentáre v režime \textit{verbose}.
\end{itemize}



\textbf{\fontfamily{lmtt}\selectfont ORFPlotter.py} generuje obraz distribúcie ORF a pomeru obsahu GC v genóme do súboru {\fontfamily{lmtt}\selectfont -ORFs.png} .
Obsahuje nasledujúce funkcie:
\begin{itemize}
  \item \textbf{\fontfamily{lmtt}\selectfont visualize(filename)} -- overuje príponu {\fontfamily{lmtt}\selectfont filname}. Funkcia zastaví vykonávanie programu v chybových prípadoch a zobrazí príslušné chybové hlásenie. Vykoná vizualizáciu súboru s anotáciami určeného genómu zobrazením ORF.
  \item \textbf{\fontfamily{lmtt}\selectfont save(outFileName, plt)} -- vytvorí adresár {\fontfamily{lmtt}\selectfont out} ak neexistuje, a uloží vygenerovaný plot {\fontfamily{lmtt}\selectfont plt} do adresára {\fontfamily{lmtt}\selectfont out}.
\end{itemize}

\textbf{\fontfamily{lmtt}\selectfont Comparison.py} vykonáva porovnanie zvolenych genómov. Percento podobnosti sa získa na základe algoritmu pairwise2.
Obsahuje nasledujúce funkcie:
\begin{itemize}
  \item \textbf{\fontfamily{lmtt}\selectfont compare(filename1, filename2, mode)} -- overuje typy súborov {\fontfamily{lmtt}\selectfont filename1} a {\fontfamily{lmtt}\selectfont filename2} a zastaví vykonávanie programu v prípade chyby a poskytne používateľovi príslušnú správu. Vykonáva porovnanie vybraných sekvencií genómu pomocou algoritmu pairwise2. Poskytuje používateľovi ďalšie komentáre v režime \textit{verbose}.
\end{itemize}

\subsubsection{\Large{Záver}}
\addcontentsline{toc}{section}{Záver}
Táto dokumentácia predstavuje komplexný prehľad softvéru, ktorý bol vyvinutý počas bakalárskej práce „Vizualizácia štruktúry genómu“.

Aplikácia pracuje v dvoch možných režimoch a umožňuje používateľovi vizualizovať a analyzovať genómy rôznych organizmov pomocou sady vopred určených techník.

Inštalácia, vykonanie, technické aspekty a scenáre použitia boli podrobne popísané v príslušných častiach.

Na záver by sa ďalšie vylepšenia mohli zamerať na aplikovánie objektovo-orientovanej paradigmy programovania na architektúru programu a na pridanie nových funkcionalít.



% zivotopis autora
%\curriculumvitae\protect
%Táto časť\/ je nepovinná. Autor tu môže uviesť\/ svoje biografické
%údaje, údaje o~záujmoch, účasti na~projektoch, účasti na~súťažiach,
%získané ocenenia, zahraničné pobyty na~praxi, domácu prax, publikácie
%a~pod.

\end{document}
